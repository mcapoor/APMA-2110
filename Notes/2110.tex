\documentclass[12pt]{report} 
\usepackage[utf8]{inputenc}
\usepackage{geometry}
\geometry{letterpaper}
\usepackage{graphicx} 
\usepackage{parskip}
\usepackage{booktabs}
\usepackage{array} 
\usepackage{paralist} 
\usepackage{verbatim}
\usepackage{subfig}
\usepackage{fancyhdr}
\usepackage{sectsty}
\usepackage[shortlabels]{enumitem}

\pagestyle{fancy}
\renewcommand{\headrulewidth}{0pt} 
\lhead{}\chead{}\rhead{}
\lfoot{}\cfoot{\thepage}\rfoot{}

%%% ToC (table of contents) APPEARANCE
\usepackage[nottoc,notlof,notlot]{tocbibind} 
\usepackage[titles,subfigure]{tocloft}
\renewcommand{\cftsecfont}{\rmfamily\mdseries\upshape}
\renewcommand{\cftsecpagefont}{\rmfamily\mdseries\upshape} %

\usepackage{amsmath}
\usepackage{amssymb}
\usepackage{mathtools}
\usepackage{empheq}
\usepackage{xcolor}

\usepackage{tikz}
\usepackage{pgfplots}
\usepackage{tikz-cd}
\pgfplotsset{compat=1.18}

\newcommand{\ans}[1]{\boxed{\text{#1}}}
\newcommand{\vecs}[1]{\langle #1\rangle}
\renewcommand{\hat}[1]{\widehat{#1}}

\renewcommand{\P}{\mathbb{P}}
\newcommand{\R}{\mathbb{R}}
\newcommand{\E}{\mathbb{E}}
\newcommand{\Z}{\mathbb{Z}}
\newcommand{\N}{\mathbb{N}}
\newcommand{\Q}{\mathbb{Q}}
\newcommand{\C}{\mathbb{C}}

\newcommand{\ind}{\mathbbm{1}}
\newcommand{\qed}{\quad \blacksquare}

\newcommand{\brak}[1]{\left\langle #1 \right\rangle}
\newcommand{\bra}[1]{\left\langle #1 \right\vert}
\newcommand{\ket}[1]{\left\vert #1 \right\rangle}

\newcommand{\abs}[1]{\left\vert #1 \right\vert}
\newcommand{\mfX}{\mathfrak{X}}
\newcommand{\ep}{\varepsilon}

\newcommand{\Ec}{\mathcal{E}}
\newcommand{\A}{\mathcal{A}}
\newcommand{\Fc}{\mathcal{F}}
\newcommand{\Cc}{\mathcal{C}}
\newcommand{\B}{\mathcal{B}}
\newcommand{\M}{\mathcal{M}}

\newcommand{\sub}{\subseteq}
\newcommand{\st}{\text{ s.t. }}
\newcommand{\card}{\text{card }}
\renewcommand{\div}{\vspace*{10pt}\hrule\vspace*{10pt}}
\newcommand{\surj}{\twoheadrightarrow}
\newcommand{\inj}{\hookrightarrow}
\newcommand{\biject}{\hookrightarrow \hspace{-8pt} \rightarrow}
\renewcommand{\bar}[1]{\overline{#1}}
\newcommand{\overcirc}[1]{\overset{\circ}{#1}}
\newcommand{\diam}{\text{diam }}


\usepackage{tcolorbox}
\tcbuselibrary{breakable, skins}
\tcbset{enhanced}
\newenvironment*{tbox}[2][gray]{
    \begin{tcolorbox}[
        parbox=false,
        colback=#1!5!white,
        colframe=#1!75!black,
        breakable,
        title={#2}
    ]}
    {\end{tcolorbox}}

\newenvironment*{exercise}[1][red]{
    \begin{tcolorbox}[
        parbox=false,
        colback=#1!5!white,
        colframe=#1!75!black,
        breakable
    ]}
    {\end{tcolorbox}}

\newenvironment*{proof}[1][blue]{
\begin{tcolorbox}[
    parbox=false,
    colback=#1!5!white,
    colframe=#1!75!black,
    breakable
]}
{\end{tcolorbox}}

\title{APMA 2110: Real Analysis}
\author{Milan Capoor}
\date{Fall 2024}

\begin{document}
\maketitle
\chapter{Analysis and Metric Spaces}
\section{Sept 05}
Some basic notation:
\begin{align*}
    \N &:= \{1, 2, 3, \ldots\} \\
    \Z &:= \{\ldots, -2, -1, 0, 1, 2, \ldots\} \\
    \Q &:= \left\{\frac{m}{n} \mid m, n \in \Z, n \neq 0\right\} \\
    \R &:= \text{the set of real numbers}\\
    \C &:= \text{the set of complex numbers}     
\end{align*}

Some basic logic:
\begin{itemize}
    \item $(A \implies B) \iff (\neg B \implies \neg A)$ (contrapositive)
    \item $E \subset X \implies \forall x \in E, \; x \in X$
\end{itemize}

    \subsection*{Sets}
    Note that in this course, $\subset$ includes the possibility of equality, while $\subsetneq$ does not.

    \textbf{Power Set:} $P(X) = \{E: E \sub X\}$

    \emph{Example:} $X = \{1, 2, 3\}$ 
    \[P(X) = \{\emptyset, \{1\}, \{2\}, \{3\}, \{1, 2\}, \{2, 3\}, \{1, 3\}, \{1, 2, 3\}\}\]

    \textbf{Sets:} Let $\E$ be a collection of sets $E$
    \begin{itemize}
        \item $\bigcup_{E \in \Ec} = \{x : x\in E, \text{ for some } E \in \Ec\}$
        \item $\bigcap_{E \in \Ec} = \{x : x\in E, \text{ for all } E \in \Ec\}$
        \item $\Ec = \{E_{\alpha} : \alpha \in A\} = \{E_{\underset{\alpha \in A}{\alpha}}\}$ 
        \item $E_{\alpha} \cap E_{\beta} = \emptyset$ for $\alpha \neq \beta$ $\iff$ $E_{\alpha}$ and $E_{\beta}$ are \emph{disjoint}
    \end{itemize}

    \textbf{Limsup and Liminf:} For $\{E_n\}_{n=1}^{\infty}$,
    \begin{align*}
        \limsup E_n &= \bigcap_{k=1}^{\infty} \bigcup_{n=k}^{\infty} E_n\\ 
        \liminf E_n &= \bigcup_{k=1}^{\infty} \bigcap_{n=k}^{\infty} E_n
    \end{align*}

    \begin{exercise}
        \textbf{Exercise:} Prove that 
        \begin{align*}
            \limsup E_n &= \{x : x\in E_n \text{ for infinitely many } n\}\\
            \liminf E_n &= \{x : x\in E_n \text{ for all but finitely many } n\}
        \end{align*}
        i.e. after first finite $n$, $x$ is in $E_n$ for all $n$.
    \end{exercise}

    \textbf{Difference and Symmetric Difference:} Let $E$ and $F$ be two sets
    \begin{align*}
        E \setminus F &= \{x : x\in E, x\not\in F\}\\
        E \triangle F &= (E \setminus F) \cup (F \setminus E)\\ 
        E^c &= X \setminus E, \; E \sub X
    \end{align*}

    \textbf{De Morgan's Laws:}
    \begin{align*}
        \left(\bigcup_{\alpha \in A} E_{\alpha}\right)^c &= \bigcap_{\alpha \in A} E_{\alpha}^c\\
        \left(\bigcap_{\alpha \in A} E_{\alpha}\right)^c &= \bigcup_{\alpha \in A} E_{\alpha}^c
    \end{align*}

    \begin{exercise}
        \textbf{Exercise:} Prove De Morgan's Laws.
    \end{exercise}

    \textbf{Cartesian Product:} If $X$ and $Y$ are sets, then $X \times Y$ is the \emph{ordered} set 
    \[X \times Y = \{(x, y): x \in X, y \in Y\}\]

    \subsection*{Relations}
    \textbf{Relations:} A \emph{relation} $R$ from $X$ to $Y$ is a subset of $X \times Y$ such that 
    \[xRy \iff (x, y) \in R\]

    \textbf{Equivalence relation:} A relation $\sim$ is an equivalence relation in the special case $Y = X$ if it is
    \begin{itemize}
        \item Reflexive: $x\sim x \quad \forall x \in X$
        \item Symmetric $x\sim y \iff y\sim x$
        \item Transitive $x\sim y, y\sim z \implies x\sim z$
    \end{itemize}

    \subsection*{Functions}
    \textbf{Mappings:} A mapping/function $f: X \to Y$ is a relation $R$ from $X$ to $Y$ such that $\forall x \in X$, there exists a \emph{unique} $y \in Y$ such that $xRy$. We write $y = f(x)$.

    \textbf{Composition:} If $f: X \to Y$ and $g: Y \to Z$, then $g \circ f: X \to Z$ is a function such that $g \circ f(x) = g(f(x))$

    \textbf{Images:} If $D \sub X, E \sub Y$, the \emph{image} of $D$ (and the \emph{inverse image}/pre-image of $E$) under $f: X \to Y$ is
    \begin{align*}
        f(D) &= \{f(x) : x \in D\}\\ 
        f^{-1}(E) &= \{x \in X : f(x) \in E\}
    \end{align*}

    For $f: X \to Y$ we further call $X$ the \emph{domain} of $f$ and $Y$ the \emph{codomain} of $f$. The \emph{range}/\emph{image} of $f$ is $f(X)$.

    \textbf{Inverses:} $f^{-1}$ defines an operation on $P(X)$ such that
    \begin{align*}
        f^{-1}\left(\bigcup_{\alpha \in A} E_{\alpha}\right) &= \bigcup_{\alpha \in A} f^{-1}(E_{\alpha})\\
        f^{-1}\left(\bigcap_{\alpha \in A} E_{\alpha}\right) &= \bigcap_{\alpha \in A} f^{-1}(E_{\alpha})\\ 
        f^{-1}(E^c) &= (f^{-1}(E))^c
    \end{align*}
    
    \begin{exercise}
        \textbf{Exercise:} Prove the above properties of inverses. Warning: in general, $f$ also commutes with unions but \emph{not} with intersections. Why?
    \end{exercise}

    \textbf{Bijectivity:}
    \begin{itemize}
        \item $f$ is \emph{injective} iff $f(x_1) = f(x_2) \implies x_1 = x_2$
        \item $f$ is \emph{surjective} iff $\forall y \in Y, \exists x \in X \st f(x) = y$
        \item $f$ is \emph{bijective} iff it is both injective and surjective
    \end{itemize}

    In the case of a bijective mapping $f$, then $f^{-1}$ is a function from $Y$ to $X$ (i.e. $f^{-1}$ has a unique value for bijective $f$)

    \subsection*{Sequences}
    \textbf{Sequences:} A sequence in a set $X$ is a function $f: \N \to X$. We $\{x_n\}$ for $x_n \in X$
    
    \textbf{Subsequence:} A subsequence $x_{n_k} \subseteq \{x_n\}$ with $n_k \in \{1, \dots, \infty\}$

    \subsection*{Ordering}
    \textbf{Partial ordering:} a partial ordering on a nonempty set $X$ is a relation $R$ on $X$ such that 
    \begin{itemize}
        \item If $xRy$ and $yRz$, then $xRz$ (transitivity)
        \item If $xRy$ and $yRx$, then $x = y$ (antisymmetry)
        \item $xRx$ for all $x$ (reflexivity)
    \end{itemize}

    \emph{Example:} Let $E$ be a set. Consider the relation $\sub$. Let $E_1, E_2, E_3 \sub E$.
    \begin{itemize}
        \item $E_1 \sub E_2$ and $E_2 \sub E_3$ implies $E_1 \sub E_3$ (transitivity $\checkmark$)
        \item $E_1 \sub E_2$ and $E_2 \sub E_1$ implies $E_1 = E_2$ (antisymmetry $\checkmark$)
        \item $E_1 \sub E_1$ (reflexivity $\checkmark$)
    \end{itemize}
    Therefore, inclusion (with equality) is a partial ordering.
    (Proof for first two by considering elements, proof for last by equality)

    \textbf{Total ordering:} A total ordering/linear ordering is a partial ordering such that for all $x, y \in X$, either $xRy$ or $yRx$.

    \emph{Example:} Inclusion is not a total ordering on $P(X)$ since (in general) $E_1 \not\sub E_2$ and $E_2 \not\sub E_1$ for $E_1 \neq E_2$.

\section{Sept 10}
    \textbf{Recall:} a \emph{partial ordering} is a relation that satisfies
    \begin{enumerate}
        \item if $xRy$ and $yRz$, then $xRz$ 
        \item if $xRy$ and $yRx$, then $x = y$
        \item $xRx$ for all $x$
    \end{enumerate}

    \emph{Examples:}
    \begin{itemize}
        \item  In the real numbers, $\leq$ is the typical ordering. 
        \item For a set $X$ and its power set $P(X)$, $\sub$ is a partial ordering.
    \end{itemize}

    \textbf{Warning:} In this class, we will use $\leq$ to denote an abstract partial ordering. 

    \textbf{Total/Linear Ordering:} A total ordering is a partial ordering such that for all $x, y \in X$, either $x\leq y$ or $y \leq x$.

    \textbf{Extrema:} If $X$ is partially ordered by $\leq$, a \emph{maximal} (resp. \emph{minimal}) element of $X$ is an element $x \in X$ such that $x \leq y \implies y = x$

    \textbf{Bounds:} If $E \sub X$, an \emph{upper} (resp. \emph{lower}) \emph{bound} for $E$ is an element $x \in X$ such that $y \leq x$ (resp. $x \leq y$) for all $y \in E$.

    \begin{tbox}{\textbf{Zorn's Lemma (transfinite induction):} If $X$ is partially ordered by $\leq$, assume every linearly ordered subset of $X$ has an upper bound. Then $X$ has a maximal element.}
        \emph{Proof:} We regard this as axiomatic
    \end{tbox}

    \textbf{Well-Ordering:} A set $X$ is \emph{well-ordered} if 
    \begin{enumerate}
        \item it is linearly ordered by $\leq$
        \item every nonempty subset of $X$ has a minimal element.
    \end{enumerate}

    \begin{tbox}{\textbf{Well-ordering Principle:} Every non-empty set $X$ can be well-ordered}
        \emph{Proof:} Consider $\mathcal{W} = \{\text{all well-ordered subsets of } X\}$.
        
        Suppose there exist well-ordered sets $E_1, E_2 \sub W$. Then each has a minimal element.  

        We know $\mathcal W$ is non-empty because for all finite subsets of $X$, we can order them (using the normal linear order on $\R$). 

        We will proceed by defining a relation $R$ between the linear orderings $\leq_1$ and $\leq_2$ of $E_1$ and $E_2$ respectively. We will say $\leq_1 R \leq_2$ if:
        \begin{enumerate}
            \item $\leq_2$ extends $\leq_1$ (i.e. $E_1 \sub E_2$ and $\leq_1 = \leq_2$ on $E_1$)
            \item If $x \notin E_1, x \in E_2$, then $y \leq_2 x$ for all $y \in E_1$
        \end{enumerate}

        \begin{exercise}
            \textbf{Exercise:} Prove that $R$ is a partial ordering in $\mathcal W$
        \end{exercise}
        
        Assume $\mathcal S = \{\leq_{\alpha} ; R\}$ is the set of linear orderings $\leq_{\alpha}$ of $E_{\alpha} \sub \mathcal W$ for $\alpha \in A$. Thus, $\leq_{\alpha} R \leq_{\beta}$ for $\alpha, \beta \in A$.

        \emph{Claim:} Let
        \[E_{\infty} = \bigcup_{\alpha \in A} E_{\alpha}\]
        equipped with the partial ordering $\leq_{\infty}$ such that $\leq_{\infty}\big\vert_{E_{\alpha}} = \leq_{\alpha}$ for all $\alpha \in A$. 

        Clearly, $\leq_{\alpha} R \leq_{\infty}$ for all $\alpha \in A$. Then for any sequence of well-ordered sets in $\mathcal W$, $E_{\infty}$ is an upper-bound. 
        
        \begin{exercise}
            \textbf{Exercise:} Verify that $\leq_{\alpha} R \leq_{\infty}$ is well defined and that $E_{\infty}$ is an upper bound for $\mathcal W$
        \end{exercise}

        By Zorn's Lemma, there exists a maximal element $E_{\max} \in \mathcal W$. (Verify it's a well-ordering by extending $\leq_{\max}$ to include any $x_0 \in X\setminus E_{\max}$ such that $x \leq x_0$ for all $x \in E_{\max}$).
        
        Consider $E_{\max} \cup \{x_0\}$. Clearly, $E_{\max} \leq E_{\max} \cup \{x_0\}$, so $E_{\max} \cup \{x_0\}$ and by the extension above, $E_{\max} \cup \{x_0\} \in \mathcal W$. This contradicts the maximality of $E_{\max}$, so $E_{\max} = X$.
    \end{tbox}

    \textbf{Definition:} Let $\prod_{\alpha \in A} X_{\alpha}$ be the set of all maps $f: A \to \bigcup_{\alpha \in A} X_{\alpha}$ such that $f(\alpha) \in X_{\alpha}$ for all $\alpha \in A$.

    \begin{tbox}{\textbf{Axiom of Choice:} If $\{X_\alpha\}_{\alpha \in A}$ is a nonempty collection of nonempty sets, $\prod_{\alpha \in A} X_{\alpha}$ is nonempty, i.e. there exists at least one choice function $f$ }
        \emph{Proof:} Let $X = \bigcup_{\alpha \in A} X_{\alpha}$. Pick a well-ordering on $X$ and $\alpha \in A$. Let $f(\alpha)$ be the minimal element of $X_{\alpha}$. Then $f \in \prod_{\alpha \in A} X_{\alpha}$ 
    \end{tbox}

    \subsection*{Cardinality} 
        \textbf{Definition:} 
        \begin{itemize}
            \item $\card X \leq \card Y$ if there exists an injective function $f: X \to Y$
            \item $\card X = \card Y$ if there exists a bijective function $f: X \to Y$
            \item $\card X \geq \card Y$ if $\card X \leq \card Y$ but $\card X \neq \card Y$ there exists a surjective function $f: X \to Y$
        \end{itemize}

        \begin{tbox}{\textbf{Property:} $\card X \leq \card Y$ iff $\card Y \geq \card X$}
            \emph{Proof:} $\card X \leq \card Y$ implies there exists an injective $f: X \to Y$. Pick $x_0 \in X$ and define $g: Y \to X$ by 
            \[g(y) = \begin{cases}
                f^{-1}(y) & y \in f(X)\\
                x_0 & \text{otherwise}
            \end{cases}\]
            
            In the first case, we have injectivity of $f$ so each $f^{-1}(y)$ is unique. In the second case we ensure surjectivity. 

            \div 

            Conversely, if $g: Y \to X$ is surjective, consider $g^{-1}(\{x\})$ for $x \in X$. These sets are non-empty and disjoint because $f$ is a map (each $x$ can map to a single $y$). Then any $f \in \prod_{x \in X} g^{-1}(\{x\})$ is an injection from $X$ to $Y$.
        \end{tbox}

\section{Sept 12}
    \begin{tbox}{\textbf{Property:} For any sets $x$ and $Y$, either $\card X \leq \card Y$ or $\card Y \leq \card X$}
        \emph{Proof Sketch:} Consider the (non-empty) set 
        \[J = \{\text{all injections } f_E: X \to Y \text{ with respect to } E \sub X\}\]
        
        Define a relation $R$ on $J$ such that $f_{E_1} R f_{e_2}$ if $E_1 \sub E_2$ and $f_{E_2}\big\vert_{E_1} = f_{E_1}$, i.e. $f_{E_2}$ is an extension of $f_{E_1}$.

        Repeating the argument of the Well-Ordering Principle, $R$ is a partial ordering. 

        Then we can find an upper bound for $J$ by considering the union of all $E \in J$ and extending the injections. 

        By Zorn's Lemma, there exists a maximal element $f_{E_{\max}} \in J$ with respect to the ordering $R$. 

        \emph{Case 1:} Suppose $E_{\max} = X$. Then $f_{E_{\max}}$ is an injection from $X$ to $Y$ so $\card X \leq \card Y$  

        \emph{Case 2:} Suppose $E_{\max} \subsetneq X$. Then $\exists x_0 \in X \setminus E_{\max}$. Consider the image $f(E_{\max})$. We claim $f(E_{\max}) = Y$ so $f_{E_{\max}}^{-1}$ is defined on all of $Y$ and is injective $Y \to X$ and we are done. Thus, it only remains to show $f(E_{\max}) = Y$. 

        If the claim is not true, $\exists y_0 \in Y$ but $y_0 \notin f_{E_{\max}}(X)$ but this is a contradiction to maximality (as in the Well-Ordering Principle proof).       
    \end{tbox}

    \begin{tbox}{\textbf{Schröder-Bernstein Theorem:} If $\card X \leq \card Y$ and $\card Y \leq \card X$, then $\card X = \card Y$}
        \emph{Note:} This seems trivial but in fact the two functions are not necessarily the same so we must construct our own bijection. 

        \emph{Proof:} Denote the cardinality injections $f: X \to Y$ and $g: Y \to X$.

        If $f(X) = Y$, then $f$ is a bijection and we are done.

        If $f(X) \neq Y$ (i.e. $f(X) \subsetneq Y$), the consider $Y_1 = Y \setminus f(X)$ and $g(Y_1)$. Then $f(Y_1) \subsetneq X$, so call $X_1 = f(Y_1)$. We now have a bijection $X_1 \to Y_1$. 

        Let's repeat. $f(X \setminus X_1) \subsetneq Y \setminus Y_1$ so define $Y_2 = (Y \setminus Y_1) \setminus f(X \setminus X_1)$.

        Now we know $f(X_1) \sub Y_2$ and $f^{-1}(Y_1) \sub X_1$ so we can define a bijection $X_2 \to Y_2$.

        Assume $X_1, \dots X_n$ and $Y_1, \dots, Y_n$ are constructed. WLOG assume that this procedure can be repeated infinitely (or else we would already have a bijection). 

        Define 
        \[\left(Y \setminus \bigcup_{i=1}^n Y_i\right) \setminus f\left(X \setminus \bigcup_{i=1}^n X_i\right) = Y_{n+1}\]
        since $f(X_i) \sub Y_{i+1}$. 

        \begin{exercise}
            \textbf{Exercise:} Verify that 
            \[g: \bigcup_{i=1}^{\infty} Y_i \to \bigcup_{i=1}^{\infty} X_i\]
            is a bijection and further that 
            \[f: \left(X \setminus \bigcup_{i=1}^{\infty} X_i\right) \to \left(Y \setminus \bigcup_{i=1}^{\infty} Y_n\right)\] 
            is also a bijection.
        \end{exercise} 

        Together, these steps show that we have a bijection on the full sets $X$ and $Y$.
    \end{tbox}

    \begin{tbox}{\textbf{Proposition}: For any set $X$, $\card X < \card P(X)$}
        \emph{Proof:} Clearly, $\forall x \in X$, we have an injection $f: X \inj P(X)$ defined by $f(x) = \{x\}$.

        We claim there is no surjection $g: X \to P(X)$ and proceed by contradiction. 

        Let $g: X \twoheadrightarrow P(X)$. Define 
        \[Y = \{x \in X \st x \notin g(x)\}\] 

        We claim $Y \notin g(X)$. If not, assume $x_0 \in X$ such that $g(x_0) = Y$. 

        \emph{Case 1:} If $x_0 \in Y$, then $x_0 \notin g(x_0) = Y$ - contradiction 

        \emph{Case 2:} If $x_0 \notin Y$, then $x_0 \in g(x_0) = Y$ - contradiction

        Therefore, $Y \notin g(X)$ so $g$ is not surjective.
    \end{tbox}

    \textbf{Countable:} A set $X$ is \emph{countably infinite} if $\card X \leq \card \N$.

    \begin{tbox}{\textbf{Proposition:} 
        \begin{enumerate}[label=(\alph*)]
            \item If $X$ and $Y$ are countable, so is $X \times Y$.
            \item If $A$ is countable and $X_{\alpha}$ is countable for every $\alpha \in A$, then $\bigcup_{\alpha \in A} X_{\alpha}$ is countable.
        \end{enumerate}}
        \emph{Proof:} 
        \begin{enumerate}[label=(\alph*)]
            \item $\card X = \card Y =\card \N$ so it suffices to show $\N \times \N = \card \N$ 
            
            $\forall n \in \N$, define $f(n) \inj (n, 1) \in \N \times \N$.

            \color{red}
            Consider $g((m, n)) \to 2^m 3^n \in \N$. Is this injective? Consider $g(m_1, n_1) = 2^{m_1} 3^{n_1}$. By the unique prime factorization of integers, $2^{m_1} 3^{n_1} = 2^m 3^n$ iff $(m_1, n_1) = (m, n)$ so $g$ is injective.

            Now we can use Schroder-Bernstein and we are done. 

            \item As $A$ is countable, $\forall \alpha \in A$, $\exists f_{\alpha}: \N \to X_{\alpha}$ So we can define $F: \N \times A \to \bigcup_{\alpha \in A} X_{\alpha}$ by 
            \[F(n, \alpha) = f_{\alpha}(n)\]
            which is surjective 

            \color{black}

        \end{enumerate}

    \end{tbox}

    \begin{tbox}{\textbf{Corollary:} $\Z$ and $\Q$ are countable}
        \emph{Proof:} $\Z = \N \cup \{-\N\} \cup 0$ 

        We can define $f: \Z^2 \to \Q$ by 
        \[f(m, n) = \begin{cases}
            \frac{m}{n} & n \neq 0\\
            0 & n = 0 
        \end{cases}\] 
    \end{tbox}

    \textbf{Convention for this course:} We will use $\R$ to denote the standard reals and will define the \emph{extended reals} $\bar R$ by $\R \cup \pm \infty$ 

    Under this notation, we can state that for any $E \sub \bar R$, $\sup \bar E$ and $\inf \bar E$ are always well-defined, i.e. all sets are bounded above by $\infty$ and below by $-\infty$.

    We define the following rules: 
    \begin{itemize}
        \item $X \pm  \infty = \pm \infty$ 
        \item $\infty + \infty = \infty$
        \item $-\infty - \infty = -\infty$
        \item $\infty - \infty$ is undefined
        \item $x(\pm \infty) = \pm \infty$ for $x > 0$ and $x(\pm \infty) = \mp \infty$ for $x < 0$
        \item $0 \cdot (\pm\infty) = 0$
    \end{itemize}

    \textbf{Note:} this last point does \emph{not} talk about limits, it is just notation 

    \begin{tbox}{\textbf{Proposition:} Every open set in $\R$ is a countable disjoint union of open intervals}
        \emph{Proof Sketch:} For all $x \in U$, there exists an open interval $I_{\alpha, \beta} = (\alpha, \beta) \sub U$ with $\alpha < x < \beta$. 

        Let $\mathcal J_x = \{x \in I_{\alpha, \beta} \; | \; I_{\alpha, \beta} \in U\}$. 

        Take $\alpha_{\inf} = \inf \mathcal \alpha$ and $\beta_{\sup} = \sup \beta$. 

        \begin{exercise}
            \textbf{Exercise:} Check that $x \in (\alpha_{\inf}, \beta_{\sup}) \sub U$
        \end{exercise}

        We call $I_x = (\alpha_{\inf}, \beta_{\sup})$ for all $x \in U$ 

        We claim $\forall x, y \in U$, either $I_x \cap I_y = \emptyset$ or $I_x = I_y$.

        Suppose $I_x \cap I_y \neq \emptyset$. Then $I_x \cup I_y$ is an open interval containing $x$, so $I_x \cup I_y \in \mathcal J_x$ but $I_x$ is maximal so this is a contradiction unless $I_x = I_y$ 

        Now we can write 
        \[U = \bigcup_{x \in U} I_x\]

        Why is this countable? We can define an injection $U \to \Q$ by choosing a rational number in each $I_x$ (exist by density of $\Q$). 
    \end{tbox}

    \subsection*{Metric Spaces}
    \textbf{Definition:} A \emph{metric space} is a set $X$ together with a \emph{distance function} $\rho: X \times X \to [0, \infty)$ such that 
    \begin{enumerate}
        \item $\rho(x, y) = 0 \iff x = y$
        \item $\rho(x, y) = \rho(y, x)$
        \item $\rho(x, y) \leq \rho(x, z) + \rho(z, y)$
    \end{enumerate}

    \textbf{Examples:} 
    \begin{itemize}
        \item $\R^n$ with $\rho(x, y) = \abs{x - y}$
        \item Set of continuous functions $f$ over $[0, 1]$ with $\rho_1(f, g) = \int0^1 \abs{f(x) - g(x)} dx$ (or alternatively $\rho_{\infty} = \sup_{0 \leq x \leq 1} \abs{f(x) - g(x)}$)
    \end{itemize}
 
    \begin{exercise}
        \textbf{Exercise:} Check the above are metric spaces
    \end{exercise}

\section{Sept 17}
\subsection*{Closed and Open Sets}  
    \textbf{Open ball:} Let $(X, \rho)$ be a metric space. If $x \in X$, $r > 0$, we define the \emph{open ball} $B(x, r) = \{y \in X \st \rho(x, y) < r\}$

    \textbf{Open set:} a set $E$ is open iff $\forall x \in E, \exists r > 0 \st B(x, r) \sub E$

    \textbf{Closed set:} a set $E$ is closed iff $E^c$ is open

    \emph{Example:} $B(x, r)$ is open. Consider $y \in B(x, r)$. Then $\rho(x, y) = s < r$. By the triangle inequality, $B(y, r - s) \sub B(x, r)$ 

    \begin{exercise}
        \textbf{Exercise:} Prove that $B(x, r)$ is open
    \end{exercise}

    \textbf{Properties:}
    \begin{itemize}
        \item $\emptyset$ is open 
        \item If $U_x$ are open sets, $\bigcup_{x \in A} U_x$ is open (as is the finite intersection)
        \item If $F_x$ are closed sets, $\bigcap_{x \in A} F_x$ is closed (as is the finite union)
    \end{itemize}

    \textbf{Interior:} Let $E \sub X$. The \emph{interior} of $E$ is 
    \[\overset{\circ}{E} = \bigcup_{O \sub E} O\]
    (this is the largest open set in $E$)

    \textbf{Closure:} The \emph{closure} of $E$ is
    \[\bar E = \bigcap_{E \sub F} F\]
    (this is the smallest closed set containing $E$)
    
    \begin{tbox}{\textbf{Proposition:} Let $(X, \rho)$ be a metric space. Let $E \sub X$ and $x \in X$. Then the following are equivalent:
        \begin{enumerate}[label=(\alph*)]
            \item $x \in \bar E$
            \item $B(x, r) \cap E \neq \emptyset$ for all $r > 0$
            \item $\exists (x_n) \sub E$ such that $x_n \to x$
        \end{enumerate}}
        \emph{Proof:} 
        ($(a) \to (b)$) Let $x \in \bar E$. Suppose $\exists r_0 >0$ such that $B(x, r) \cap E = \emptyset$. Then $E \sub (B(x, r_0))^c$. But $(B(x, r_0))^c$ is closed so $\bar E \sub (B(x, r_0))^c$ so $x \in B(x, r_0) \sub (\bar E)^c$ but this implies $x \in (\bar E)^c$ which is a contradiction. 

        ($(b) \to (c)$) Let $r = \frac{1}{n}$. By (b), $B(x, \frac{1}{n}) \cap E \neq \emptyset$. Choose $x_n \in B(x, \frac{1}{n}) \cap E$. Certainly $\rho(x_n, x) < \frac{1}{n}$ so $\lim \rho(x_n, x) = 0$ and $x_n \to x$

        ($(c) \to (a)$) If $x \notin \bar E$, $x \in (\bar E)^c$ but $(\bar E)^c$ is open so $\exists r > 0 \st B(x, r) \sub (\bar E)^c \sub E^c$. Then there cannot exist any sequence in $E$. But this contradicts $x_n \to x$
    \end{tbox}

\subsection*{Density}
    \textbf{Dense:} $E$ is dense in $X$ if $\bar E = X$ (examples $\R^n, \Q^n$)

    \textbf{Nowhere dense:} $E$ is nowhere dense if $(\bar E)^{\circ} = \emptyset$ (example: emptyset)

    \textbf{Separable:} $X$ is separable if there exists a countable dense subset $E \sub X$

    \textbf{Limits:} In this class, $x_n \to x$ iff $\lim_{n\to\infty} \rho(x_n, x) = 0$

\subsection*{Continuity} 
    Let $\mathcal C = \{\text{continuous functions on } [0,1]\}$.

    \textbf{Continuity at a point:} If $(X_1, \rho_1)$ and $(X_2, \rho_2)$ are metric spaces, $f: X_1 \to X_2$ is continuous at $x \in X_1$ if $\forall \ep > 0, \exists \delta_x > 0$ such that $\forall y \in X_1$ such that $\rho_1(x, y) < \delta_x$ (i.e. $y \in B_1(x, \delta_x)$),  
    \[\rho_2(f(x), f(y)) < \ep\]
    (i.e. $f(y) \in B_2(f(x), \ep)$)

    \textbf{Continuity on a set:} $f$ is continuous in $X$ iff $f$ is continuous at every $x \in X$

    \textbf{Uniform Continuity:} $f$ is uniformly continuous if $\delta$ is independent of $x$, i.e. $\forall \ep > 0$, $\exists \delta > 0$ such that 
    \[\rho_1(x, y) < \delta \implies \rho_2(f(x), f(y)) < \ep\]
    for all $x \in X$. 

    \begin{tbox}{\textbf{Proposition:} $f: X_1 \to X_2$ is continuous iff $f^{-1}(U) \sub X_1$ is open for all open $U \sub X_2$}
        \emph{Proof:} Let $f$ be continuous and $U \sub X_2$ be open. $f^{-1}(U) = \emptyset$ is open so take $x \in f^{-1}(U)$. Then $f(x) = y \in U$. 
        
        Since $U$ is open, $\exists \ep_y > 0 \st B_2(y, \ep_y) = B_2(f(x), \ep_y) \sub U$. 
        
        By continuity, $\exists \delta_x > 0$ such that $\forall z \in B_1(x, \delta_2)$, 
        \[\rho_2(f(x), f(z)) < \ep_y \implies f(z) \in B_2(y, \ep_y) \sub U \implies z \in f^{-1}(U)\]
        so $B_1(x_1, \delta_x) \sub f^{-1}(U)$ and $f^{-1}(U)$ is open.

        Conversely, suppose $f^{-1}(U)$ is open for all open $U \sub X_2$. Let $\ep > 0$. Consider $y = f(x) X_2$. Then $B_2(y, \ep)$ is open so $f^{-1}(B_2(y, \ep))$ is open by assumption. 
        
        Let $x \in f^{-1}(B_2(y, \ep))$. Then $\exists \delta_x$ such that $B_1(x, \delta_x) \sub f^{-1}(B_2(y, \ep))$. 

        Then $f(B_1(x, \delta_x)) \sub B_2(y, \ep)$ which is precisely the definition of continuity.
    \end{tbox}

\subsection*{Cauchy Sequences}
    \textbf{Cauchy Sequence:} A sequence $(x_n)$ in a metric space $(X, \rho)$ is Cauchy if $\forall \ep > 0, \exists N \in \N$ such that $\forall m, n \geq N$,
    \[\rho(x_m, x_n) < \ep\]

    \textbf{Completeness:} A subset $E \sub X$ is \emph{complete} if every Cauchy sequence $x_n \in E$ has a limit $x \in E$

    \emph{Examples:} 
    \begin{itemize}
        \item In $\R^n$, any bounded closed subset is complete. 
        \item $(\mathcal C, \rho_{\infty})$ is complete 
    \end{itemize}

    \begin{exercise}
        \textbf{Exercise:} Prove that $(\mathcal C, \rho_{\infty})$ is complete for 
        \[\rho_{\infty}(x, y) = \sup_{x \in [0,1]} \abs{f(x) - g(x)}\] 
        (though in general this is not true for other metrics)
    \end{exercise}
    
    \begin{tbox}{\textbf{Proposition:} A closed subset $(X, \rho)$ of a complete metric space is complete and complete subsets of a metric space must be closed}
        \emph{Proof:} 
        
        \begin{exercise}
            \textbf{Exercise}
        \end{exercise}
    \end{tbox}

    \textbf{Set Distance:}
    \begin{itemize}
        \item Let $x \in X$ and $E \sub X$. The \emph{distance} from $x$ to $E$ is
        \[\rho(x, E) = \inf\{\rho(x, y): y \in E\}\] 
        \item For $E, F \sub X$, 
        \[\rho(E, F) = \inf\{\rho(x, y): x \in E, y \in F\}\]
    \end{itemize}

    \textbf{Diameter:} $\diam E = \sup\{\rho(x, y): x, y \in E\}$

    \textbf{Bounded:} $E$ is bounded iff $\diam E < \infty$

    \textbf{Open cover:} Let $\{V_{\alpha}\}_{\alpha \in A}$ be a family of sets. $\{V_{\alpha}\}$ \emph{covers} $E$ if 
    \[E \sub \bigcup_{\alpha \in A} V_{\alpha}\]

    \textbf{Total boundedness:} $E$ is \emph{totally bounded} if $\forall \ep > 0$, $E$ can be covered by finitely many balls of radius $\ep$

    \emph{Example:} $\R^n$ is totally bounded. \emph{Proof:} consider a hypercube of side length $R$. Clearly we can divide this into $\ep$-cubes and then take slightly larger balls to cover the whole space.

    \begin{tbox}{\textbf{Theorem (Characterization of Compactness):} The following are equivalent: 
        \begin{enumerate}
            \item $E$ is complete and totally bounded
            \item Every sequence in $E$ has a convergent subsequence with its limit in $E$
            \item If $\{V_{\alpha}\}_{\alpha \in A}$ is an open cover of $E$, then there exists a finite set $F \sub A$ such that $\{U_{\alpha}\}_{\alpha \in F}$ covers $E$
        \end{enumerate}}
        \emph{Proof:} HW
    \end{tbox}

\section{Sept 19}
    \textbf{Products of Metric Spaces:} Let $(X, \rho_1)$ and $(Y, \rho_2)$ be metric spaces. Define the \emph{product metric} on $X \times Y$ by $(X_1 \times X_2, \rho_1 \times \rho_2)$ where
    \[\rho_1 \times \rho_2 = \sqrt{\rho_1^2(x_1, y_1) + \rho^2(x_2, y_2)}\]
    (so called \emph{Euclidean Metric})

    Though many other metrics are possible, such as $\max(\rho_1, \rho_2)$ and $\rho_1 + \rho_2$. 

    In general, we will simply take the Euclidean metric because all these metrics are equivalent in the sense that $\exists C_1, C_2$ such that 
    \[C_1(\rho_1 \times \rho_2)_1 \leq C_2(\rho_1 \times \rho_2)_2 \leq C_2(\rho_1 \times \rho_2)_3\]

    \emph{Properties:}
    \begin{itemize}
        \item $\rho_1 \times \rho_2 \to 0 \iff \rho_1 \to 0$ and $\rho_2 \to 0$
    \end{itemize}

\chapter{Measure Theory}
\section{Sept 19}
    \subsection*{Measure Theory Motivation}
        \textbf{Riemann Integral:} Let $f: [a, b] \to \R$. We subdivide $[a, b]$ by 
        \[a = x_0 < x_1 < \dots < x_n = b\]
        and define subintervals $[x_i, x_{i+1}]$. 

        Then 
        \[\int f(x) \;dx = \lim_{n \to \infty}\sum_{i = 1}^{n} f(x_i)\cdot (x_{i+1} - x_i)\]

        \textbf{Convergence:} Many times, we are interested in the question: 
        \[\lim_{n \to \infty} \int_0^1 f_n(x)\; dx \overset{?}{=} \int_0^1 f(x)\; dx\]
        for $f_n(x) \to f(x)$. 

        This is easy when $f_n \to f$ uniformly but in general, we need something else. 

        In Riemann integration, we divide the domain into intervals and sum the function over these intervals.

        In Lebesgue intergration, we instead divide \emph{the range}, i.e. we take a set 
        \[E_i = \{x: a_n \leq f(x)\leq a_{n+1}\}\] 

        \textbf{Measure:} We define $\mu(E)$, the \emph{measure} of a subset, by:
        \begin{enumerate}
            \item (Countable Additivity) $\{E_n\}$ such that $E_i \cap E_j = \emptyset$ for $i \neq j$ then $\mu(\bigcup_{n=1}^{\infty} E_n) = \sum_{i=1}^n \mu(E_n)$
            \item (Translation invariance) $\mu(E + r) = \mu(\{x + r: x \in E\})= \mu(E)$
            \item $\mu([0,1]) = 1$
        \end{enumerate} 

        \begin{tbox}{\textbf{Proposition:} There is no measure $\mu$ satisfying the above properties which is defined for all subsets of $[0, 1)$}
            \emph{Proof:} Step 1. Let $\Q_1 = \Q \cap [0, 1)$. Define an equivalence relation $x \sim y$ iff $x - y \in \Q_1$.

            Now consider the equivalence class $\mathcal E_x = \{y \in [0, 1): y \sim x\}$. (As it is an equivalence class: $\Ec_x \cap \Ec_y \neq \emptyset \implies \Ec_x = \Ec_y$)

            And clearly, 
            \[[0, 1) = \bigcup_{x \in [0, 1)} \Ec_x\]

            By the Axiom of Choice, choose a unique element $e_x \in \Ec_x$. Define $N = \{e_x\}$. Now $e_x - e_y \notin \Q_1$. 

            Step 2. $\forall r \in \Q_1$, define 
            \[N_r = \{e_x + r: e_x \in N \cap [0, 1 - r)\} \cup \{e_x + r - 1, e_x \in N \cap [1 - r, 1]\}\] 
            (the first set is the points that don't leave the interval under translation, the second set is the pullback of the points that do) 

            Step 3. We claim 
            \[[0, 1) = \bigcup N_r, \; N_r \cap N_s = \emptyset \text{ for } r \neq s\]

            \emph{Proof:} 
            \begin{enumerate}
                \item (Subset) $\forall y \in [0, 1)$, $\exists e_x \in N$ such that $y - e_x \in \Q_1$. 
                
                If $y \geq e_x$, $r = e_x - y +1$. Otherwise, $r = e_x - y$. 

                \item (Disjoint Union) Suppose $N_r \cap N_s \neq \emptyset$. Let $r \neq s$. Select $y \in N_r \cap N_s$ so $y - s \in N$ and $y - r \in N$
                
                Case 1. $y - s \neq y - r$. But then 
                \[(y - r) - (y - s) = s - r \in \Q_1\]
                which is a contradiction of the construction of $N$.

                Case 2. $y - s \neq y - r + 1$. Contradiction again by rational difference. 
            \end{enumerate}

            Step 4. By the definition of a measure,
            \begin{align*}
                \mu(N_r) &= \mu(N_r \cap (0, 1 -r)) + \mu(N_r \cap [1 - r, 1])\\ 
                    &= \mu(N)
            \end{align*}
            \begin{exercise}
                \textbf{Exercise:} Check that $\mu(N_r) = \mu(N)$
            \end{exercise}

            But by countable Additivity, 
            \[1 = \mu([0, 1)) = \sum_{r \in \Q_1}^{\infty} \mu(N_r) = \begin{cases}
                0\\ \infty
            \end{cases}\] 
            which is a contradiction.  
        \end{tbox}

        \textbf{Conclusion:} it is not always possible to define a measure so we need to be careful. 

    \subsection*{Algebras}
        \textbf{Algebra:} Given a set $X$, an \emph{algebra} is a collection of subsets $\A \sub P(X)$ such that if $E_1, \dots, E_n \sub \A$, 
        \begin{enumerate}
            \item $\bigcup_{i=1}^n E_i \in \A$
            \item $E \in \A \implies E^c \in \A$
        \end{enumerate}

        Property 2 gives us that $X \in \A$ and $\emptyset \in \A$ ($E \cup E^c = X, \; X^c = \emptyset$)

        \textbf{Sigma Algebra:} An algebra $\A$ is a \emph{$\sigma$-algebra} if it is closed under countable unions and complements, i.e. for $E_1, E_2, \dots \in \A$,
        \begin{enumerate}
            \item $\bigcup_{i=1}^{\infty} E_i \in \A$
            \item $E \in \A \implies E^c \in \A$
        \end{enumerate}

        \emph{Remark:} It suffices to demand closure for disjoint countable unions since 
        \[\bigcup_{n=1}^\infty E_i = \bigcup_{n=1}^\infty F_i\] 
        for $F_k = E_k \setminus \bigcup_{i=1}^{k-1} E_i$ and $F_i \cap F_{i+1} = \emptyset$ 

        \emph{Examples:}
        \begin{itemize}
            \item $P(X)$
            \item $\phi$, $X$
            \item $\A = \{E \sub X: E \text{ countable or } E^c \text{ countable}\}$
        \end{itemize}

        \begin{tbox}{\textbf{Proposition:} Let $\A_1, \A_2$ be two $\sigma$-algebras on $X$. Then $\A_1 \cap \A_2$ is also a $\sigma$-algebra}
            \begin{exercise}
                \textbf{Exercise:} Prove this proposition (easy using definition)
            \end{exercise}
        \end{tbox}

        \textbf{Generated $\sigma$-algebra:} Given a collection of subsets $\Ec \sub P(X)$, there exists a smallest $\sigma$-algebra containing $\Ec$, denoted 
        \[M(\Ec) = \bigcap_{\Ec \sub \A} \A\]

        \begin{tbox}{\textbf{Lemma:} $\Ec \sub M(\Fc) \implies M(\Ec) \sub M(\Fc)$}
            \emph{Proof:} Omitted
        \end{tbox}

    \subsection*{Metric Spaces}
        \textbf{Borel $\sigma$-algebra:} Let $(X, \rho)$ be a metric space. We call the $\sigma$-algebra generated by the open sets of $X$, the \emph{Borel $\sigma$-algebra} $B_x$ on $X$. 
    
        This is a $\sigma$-algebra because $X, \emptyset, \bigcup_{i=A} U_i$ are all open since their union is open and we have complements from the generating set. 

        We define 
        \begin{align*}
            \bigcup_{n=1}^\infty F_n &= F_{\sigma}\\ 
            \bigcap_{n=1}^\infty O_n &= G_{\delta}
        \end{align*}
        for $F_n$ closed and $O_n$ open. 

        \textbf{Example:} The Borel set of $\R$, $B_{\R}$ can be generated by any of the following:
        \begin{enumerate}
            \item open intervals $\Ec_1 = \{(a, b) : a < b\}$
            \item closed intervals $\Ec_2 = \{[a, b]: a < b\}$ 
            \item the half-open intervals $\Ec_3 = \{(a, b]: a < b\}$, $\Ec_4 = \{[a, b): a < b\}$
            \item open rays $\Ec_5 = \{(a, \infty): a \in \R\}$, $\Ec_6 = \{(-\infty, a): a \in \R\}$
            \item closed rays 
        \end{enumerate}

        \begin{exercise}
            \textbf{Exercise:} 
            \begin{enumerate}
                \item Prove that $(a, b] = \bigcap_{n=1}^{\infty} (a, b + \frac{1}{n})$ and $(a, b) = \bigcup_{n=1}^{\infty} [a + \frac{1}{n}, b - \frac{1}{n}]$
                \item Prove that the above methods all generate $B_{\R}$ 
            \end{enumerate}
        \end{exercise}

        \textbf{Conclusion:} any open set in $\R$ is the countable union of open intervals

\section{Sept 24}
    Recall last time, we were trying to characterization the Borel $\sigma$-algebra of $\R$, $\B_{\R}$. 

    
    \begin{tbox}{\textbf{Proposition:} We claim that $\B_{\R}$ is generated by:
        \begin{enumerate}
            \item open intervals $\Ec_1 = \{(a, b) : a < b\}$
            \item closed intervals $\Ec_2 = \{[a, b]: a < b\}$ 
            \item the half-open intervals $\Ec_3 = \{(a, b]: a < b\}$, $\Ec_4 = \{[a, b): a < b\}$
            \item open rays $\Ec_5 = \{(a, \infty): a \in \R\}$, $\Ec_6 = \{(-\infty, a): a \in \R\}$
            \item closed rays 
        \end{enumerate}}
        \emph{Proof:} 
        \begin{enumerate}
            \item Open intervals. 
            
            Let $\E_1 = \{(a, b): a < b\}$. Clearly $B_{E_1} \sub B_{\R}$ because any open set $O \sub \B_{\R}$. 

            For the other direction, we also have 
            \[O = \bigcup_{i=1}^{\infty} (a_i, b_i)\]
            (a countable union), so $B_{\R} \sub B_{\Ec_1}$ 

            \item Closed intervals.
            
            We claim 
            \[(a, b) = \bigcup_{n=1}^{N} [a + \frac{1}{n}, b - \frac{1}{n}]\]
            for $N$ sufficiently large.

            \begin{exercise}
                \emph{Proof:} HW
            \end{exercise}

            Now $\forall y \in (a, b)$, 
            \[y \in \bigcup_{n = 1}^N [a + \frac{1}{n}, \frac{b}{\frac{1}{n}}] \implies a < y < b\] 
            for $N$ sufficiently large.

            For the other direction, take $[a, b] \in \Ec_2$. Then 
            \[[a, b] = \bigcap_{n=1}^{\infty} (a - \frac{1}{n}, b + \frac{1}{n})\]

            \begin{proof}
                \emph{Proof:} $[a, b] \sub \bigcap_{n=1}^{\infty} (a - \frac{1}{n}, b + \frac{1}{n})$ is clear.

                For the other direction, let $y in \bigcap_{n=1}^{\infty} (a - \frac{1}{n}, b + \frac{1}{n})$. Suppose $a \leq y \leq b$ is false. Then $y \notin (a - \frac{1}{N}, b + \frac{1}{N})$ so it cannot be in the intersection
            \end{proof}

            \begin{exercise}
                \textbf{Exercise:} Prove the last two versions: half intervals and rays
            \end{exercise}
        \end{enumerate}
    \end{tbox}

    \textbf{Recall:} For a cartesian product $X_1 \times X_2 \times \dots \times X_n$ of metric spaces with $(X_i, \rho_i)$, we define the \emph{product metric} by ($X_1 \times X_2 \times \dots \times X_n, \rho)$ where 
    \[\rho(\bar x, \bar y) = \sqrt{\rho_1^2(x_1, y_1) + \dots + \rho_n^2(x_n, y_n)}\]
    where $\bar x = (x_1, x_2, \dots, x_n)$ with $x_i \in X_i$ (and similarly for $\bar y$)

    \begin{tbox}{\textbf{Proposition:} 
        \[\lim_{m \to \infty} \rho(\bar x, \bar y) = 0 \iff \lim_{m \to \infty} \rho_i(x_i^m, y_i^m) = 0\]}
        \emph{Proof:} Omitted
    \end{tbox}

    In this way, we can consider $\R^n$ as a metric space with this Euclidean metric. What is the Borel set of $\R^n$?

    \begin{tbox}{\textbf{Proposition:} $B_{\R^n}$ is  }
        \emph{Proof:} First take $O_i$ open set in $X_i$ 
        \[\bigoplus_{i=1}^n O_i = O_1 \times O_2 \times \dots \times O_n \]

        We claim that this is an open set in the $X_1 \times X_2 \times \dots \times X_n$ topology.

        \begin{proof}
            \emph{Proof:} Take $\bar x \in \bigoplus_{i=1}^n O_i$ with $x_i \in O_i$. 

            It suffices to show $\exists \ep_0 > 0$ such that $B_{\ep_0}(\bar x) \sub \bigoplus_{i=1}^n O_i$ where 
            \[B_{\ep_0}(\bar x) = \{\bar y: \rho(\bar x, \bar y) < \ep_0\}\]
            so $\bar y \in B_{\ep}(\bar x)$ iff $\rho_i(x_i, y_i) < \ep_0$ for all $i$.

            Hence $y_i \in B_{\ep_0}(x_i) \sub O_i$
        \end{proof}

        Let $\bigotimes_{i=1}^n \B_{x_i}$ be the Borel set generated by $\bigoplus_{i=1}^n O_i$

        Clearly, $\bigoplus_{i=1}^n \B_{x_i} \sub \B_{x_1 \times x_2 \times \dots \times x_n}$
        
        \textbf{Lemma:} If $x_i$ is separable then 
        \[\bigotimes_{i=1}^n B_{x_i} = \B_{x_1 \times x_2 \times \dots\times x_n}\]

        In particular: 
        \[\bigotimes_{i=1}^n \B_{\R} = \B_{\R^n}\]

        \begin{proof}
            \emph{Proof:} It suffices to show that $\forall \bar x, \ep$, 
            \[\B_{\ep}(\bar x, \ep) \sub \bigotimes_{i=1}^n B_{x_i}\]

            Let $\mathcal{C}_i$ be a countable subset of $X_i$ such that $\bar{\mathcal{C}_i} = X_i$ for all $1 \leq i \leq n$

            We claim 
            \[B_{\ep}(\bar x) \sub \bigcup_{\begin{subarray}{c} c_i \in \mathcal{C}_i \\ r_i \in \Q \end{subarray}} \bigotimes_{i=1}^n B_{r_i}(c_i)\]
            for $\sqrt{r_1^2 + \dots + r_n^2} < \ep$

            (And this has cardinality $\N^{2n}$ so countable)

            Pick 
            \[\bar y \in B_{\ep}(\bar x) \sub \bigcup_{\begin{subarray}{c} c_i \in \mathcal{C}_i \\ r_i \in \Q \end{subarray}} \bigotimes_{i=1}^n B_{r_i}(c_i) \sub \bigotimes_{i=1}^n \B_{x_i}\] 

            Then 
            \[\sigma(\bar x, \bar y) = \sqrt{\rho_1^2(y_1, x_2), \dots, \rho_n^2(y_n, x_n)} < \ep\]
            but each $\rho_i^2(y_i, x_i$) is fixed so we may choose $c_i \in \mathcal C, r_i \in \Q$ such that 
            \[\rho_i(y_i, c_i) < r_i = \rho_i(y_i, x_i) - [\rho(y_i, x_i) - \rho(y_i, c_i)]\]
            by density (from separability) 


        \end{proof}

        Since $\Q^n \sub \R^n$ which is countable and dense, $\R^n$ is separable and we are done. 
    \end{tbox}

    \subsection*{Measure Spaces}
        Recall that we could not always define a measure except on a $\sigma$-algebra. Therefore, we limit our attention.

        \textbf{Measure space:} $(X, \M)$ where $X$ is a set and $\M$, a $\sigma$-algebra, is the ``measureable sets''

        \textbf{Measure:} For a measure space $(X, \M)$, we define a \emph{measure} $\mu: \M \to [0, \infty]$ such that 
        \begin{enumerate}
            \item $\mu(\emptyset) = 0$
            \item (Countable additivity) if $\{E_j\}_1^{\infty}$ is a sequence of pairwise disjoint sets in $\M$, then 
            \[\mu\left(\bigcup_{j=1}^{\infty} E_j\right) = \sum_{j=1}^{\infty} \mu(E_j)\] 
        \end{enumerate}

        Intuitively, this countable additivity property lets us pull out the limits:
        \[\mu\left(\lim_{n \to \infty} \bigcup_{1}^n E_j\right) = \lim_{n\to \infty} \sum_{i=1}^n \mu(E_j)\] 

        \textbf{$\sigma$-finite:} If $\mu(X) = \infty$ but 
        \[X = \bigcup_{i=1}^{\infty} X_i\]
        where $\mu(X_i) < \infty$ for all $i$, then we call $X$ \emph{$\sigma$-finite}

        \textbf{Example:} Let $(X, P(X))$ be a measure space. Let $f: X \to [0, \infty]$. For each $E \in P(X)$, we define 
        \[\mu(E) = \sum_{x \in E} f(x) = \sup\{\sum_{x \in F} f(x): F \sub E \land F \text{ finite}\}\] 

        \begin{exercise}
            \textbf{Exercise:} Prove that $\mu$ is a measure on $P(X)$
        \end{exercise}

        In particular:
        \begin{itemize}
            \item $f(x) = 1$ for all $x$, then $\mu(E)$ is the \emph{counting measure}
            \item Take $x_0 \in X$ and define 
            \[f(x) = \begin{cases}
                1 & x = x_0\\
                0 & x \neq x_0
            \end{cases}\] 
            is the \emph{Dirac-Delta Mass} at $x_0$ 
        \end{itemize}

        \textbf{Example:} Let $X$ be an uncountable set. Let $\M = \{E \text{ is finite or } E^c \text{ is finite}\}$
        
        Define 
        \[\mu(E) = \begin{cases}
            0 & E \text{ is countable}\\
            1 & E^c \text{ is countable}
        \end{cases}\]

        \begin{exercise}
            \textbf{Exercise:} Check that $\M$ is a $\sigma$-algebra and that $\mu$ is a measure
        \end{exercise}

\section{Sept 26}
        \begin{tbox}{\textbf{Theorem (Properties of Measures)}: Let $(X, \M, \mu)$ be a measure space. Then:
            \begin{enumerate}
                \item (Monotonicity) with $E \sub F$ with $E, F \in \M$, then 
                \[\mu(E) \leq \mu(F)\]
                \item (Subadditivity) If $\{E_j\}_{1}^{\infty} \in \M$, then 
                \[\mu\left(\bigcup_{j=1}^{\infty} E_j\right) \leq \sum_{j=1}^{\infty} \mu(E_j)\]
                \item (Continuity from Below) If $\{E_j\}_1^{\infty} \sub \M$ and $E_1 \sub E_2 \sub \dots$, then
                \[\mu\left(\bigcup_{j=1}^{\infty} E_j\right) = \lim_{j \to \infty} \mu(E_j)\] 
                \item (Continuity from Above) If $\{E_j\}_1^{\infty} \sub \M$ and $E_1 \supset E_2 \supset \dots$ and $\mu(E_1) < \infty$, then 
                \[\mu\left(\bigcap_{j=1}^{\infty} E_j\right) = \lim_{j \to \infty} \mu(E_j)\]
            \end{enumerate}}
            \emph{Proof:} 
            (1) Take $E \sub F \in \M$. We want to use finite additivity. Consider $F \setminus E = F \cap E^c$. Certainly, $E \cap (F \cap E^c) = \emptyset$ so
            \[\mu(F) = \mu(E \cup F \setminus E) = \mu(E) + \mu(F \setminus E)\]
            but the measure is nonnegative so $\mu(E) \leq \mu(F)$. 

            And in fact, if $\mu(F) < \infty$, then $\mu(F) - \mu(E) = \mu(F \setminus E)$.
            
            \div 

            (2) Once again, we would like to take advantage of finite additivity by expressing $\bigcup_{j=1}^{\infty} E_j$ as a countable disjoint union.

            Let 
            \[F_k = E_k \setminus \bigcup_{i=1}^{k - 1} E_i = \bigcup_{j=1}^{\infty} F_k \]
            so 
            \[\mu\left(\bigcup_{j=1}^{\infty} E_j\right) = \mu\left(\bigcup_{j=1}^{\infty} F_j\right) = \sum_{j=1}^{\infty} \mu(F_j) \leq \sum_{j=1}^{\infty} \mu(E_j)\]

            \div 

            (3) Let $E_1 \sub E_2 \sub \dots \sub E_n \sub \dots$. Denote $E_0 = \emptyset$. Then 
            \begin{align*}
                E_1 &= E_1 \setminus \emptyset\\
                E_2 &= E_1 \cup (E_2 \setminus E_1)\\
                E_3 &= E_2 \cup (E_3 \setminus E_2)\\
                &= E_1 \cup (E_2 \setminus E_1) \cup (E_3 \setminus E_2)\\
                &= (E_1 \setminus \emptyset) \cup (E_2 \setminus E_1) \cup (E_3 \setminus E_2)\\
            \end{align*}
            and each of these sets are disjoint. 

            Inductively define 
            \[E_n = \bigcup_{k=0}^{n-1} E_{k+1}\setminus E_{k}\]

            We claim 
            \[\bigcup_{n=1}^{\infty} E_n = \bigcup_{n=0}^{\infty} (E_n \setminus E_{n-1})\]

            By additivity, 
            \begin{align*}
                \mu\left(\bigcup_{i=1}^\infty E_n\right) &= \sum_{n=0}^{\infty} \mu(E_{n+1} \setminus E_{n})
                    &= \lim_{N \to \infty} \sum_{n=0}^N \mu(E_{n+1} \setminus E_{n})\\ 
                    &=  \lim_{n \to \infty} \mu(E_n)
            \end{align*}

            \div 

            (4) Let $E_1 \supset E_2 \supset \dots$ and $\mu(E_1) < \infty$. Define $F_j = E_1 \setminus E_j$. 

            Clearly, $F_n \sub F_{n+1}$. By part 3, 
            \[\mu\left(\bigcup_{n=1}^\infty F_n\right) = \lim_{n\to \infty} \mu(F_n)\]
            and 
            \[F_j = E_1 \setminus E_j \implies \bigcup_{n=1}^n F_j = E_1 \setminus \bigcup_{n=1}^n E_j\]
            (by $E_1 \supset E_2 \supset \dots$)

            So 
            \[\mu\left(E_1 \setminus \bigcap_{j=1}^{\infty}\right) = \lim_{n\to \infty} \mu(E_1 \setminus \bigcap_{j=1}^n E_j)\]

            By Part 1, 
            \begin{align*}
                \lim_{n\to \infty} \mu(E_1 \setminus \bigcap_{j=1}^n E_j) &= \lim_{n \to \infty} \left[\mu(E_1) - \mu(\bigcap_{j=1}^n E_j)\right]\\ 
                    &= \lim_{n \to \infty} \left[\mu(E_1) - \mu(E_n)\right]\\ 
                    &= \lim_{n \to \infty} \mu(E_n)
            \end{align*}
        \end{tbox}

    \subsection*{Constructing Measures}
        We have showed that finding measures is hard in general. Let's construct them instead. 

        \textbf{Outer Measure:} Let $X$ be a set and $\mu^*: P(X) \to [0, \infty]$ be an outer measure if
        \begin{enumerate}
            \item $\mu^*(\emptyset) = 0$
            \item (Monotonicity) $E \sub F \implies \mu^*(E) \leq \mu^*(F)$
            \item (Subadditivity) $\mu^*\left(\bigcup_{j=1}^{\infty} E_j\right) \leq \sum_{j=1}^{\infty} \mu^*(E_j)$
        \end{enumerate}
        (note: this is \emph{almost} a measure and would be if we allowed additivity rather than subadditivity)

        \begin{tbox}{\textbf{Proposition:} Let $\Ec \sub P(X)$ such that $X, \emptyset \in \Ec$. Define $\rho: \Ec \to [0, \infty]$ with $\rho(\emptyset) = 0$. $\forall A \sub P(X)$, let
            \[\mu^*(A) = \inf\left\{\sum_{i=0}^\infty \rho(E_j) \bigg\vert E_j \in \Ec, \; A \sub \bigcup_{j=1}^\infty E_j\right\}\]
            (i.e. take the inf of the sum of all coverings of $A$). Then $\mu^*$ is an outer measure.}
            
            \emph{Proof:}
            
            First note that $\mu^*$ is well-defined: certainly $A \sub X$ so the set will not be empty and the inf is well defined. 

            Clearly, $\mu^*(\emptyset) = 0$ because $\rho(\emptyset) = 0$.

            (Monotonicity) Let $A \sub B$ and $\{E_j \in \Ec\}_1^{\infty}$ be any covering of $B$. Since $A \sub B$, 
            \[\mu^*(A) \leq \sum_{n=0}^\infty \rho(E_j)\]

            Taking the inf, 
            \[\mu^*(A) \leq \inf\left\{\sum_{i=1}^{\infty} \rho(E_j)\right\} \sub \mu^*(B)\]

            (Subadditivty) Take $\bigcup_{j=1}^{\infty} A_j$ for all $A_j \in P(X)$. 

            By definition of $\inf$, $\forall \ep > 0$ there exists $E_{jk} \sub \Ec$ such that  
            \[\sum_{i=1}^{\infty} \rho(E_{jk}) \leq \mu^*(A_j) + \frac{\ep}{2^j}\]
            so 
            \[\bigcup_{j=1}^{\infty} A_j \sub \bigcup_{jk}^{\infty} E_{jk}\]
            for $E_{jk} \in \Ec$. 

            Then 
            \begin{align*}
                \mu^*\left(\bigcup_{j=1}^\infty A_j\right) &\leq \sum_{j, k}^\infty  \rho(E_{jk})\\ 
                    &\leq \sum_{j=1}^{\infty} \mu^*(A_j) + \ep
            \end{align*}

            Then certainly, 
            \[\mu^*\left(\bigcup_{j=1}^\infty A_j\right) \leq \sum_{j=1}^\infty \mu^*(A_j)\]
        \end{tbox}

        \textbf{$\mu^*$-measurable (Carathéodory Criterion):} a collection $\M$ of subsets of $X$ is $\mu^*$-measurable if, given $A \in \M$, for all $E \sub P(X)$,
        \[\mu^*(E) = \mu^*(E \cap A) + \mu^*(E \cap A^c)\]

        And in fact, it suffices to show 
        \[\mu^*(E) \geq \mu^*(A \cup E) + \mu^*(E\cap A^c)\] 


\section{Oct 01}

    \begin{tbox}{\textbf{Carathéodory Procedure:} If $\mu^*$ is an outer measure and $\M$ are $\mu^*$-measurable sets, then $\M$ is a $\sigma$-algebra and $\mu^*\big\vert_{\M}$ is a measure on $\M$}
        
        \emph{Proof:} 

        STEP 1. $A \in \M$, $A^c \in \M$ by definition

        \div 

        STEP 2. Let $A, B \in \M$ and $A \cup B \in \M$. Then 
        \[\mu^*(A \cup B) = \mu^*(A) + \mu^*(B), \; A \cap B = \emptyset\]

        Since $\mu^* < \infty$ by definition, it suffices to show 
        \begin{align*}
            \mu^*(E) &\geq \mu^*(E \cap A) + \mu^*(E \cap A^c)\\ 
            &\geq \mu^*(E \cap (A \cup B)) + \mu^*(E \cap (A \cup B)^c)
        \end{align*}

        Since $E \in \M$ satisfies the Carathéodory Criterion (by assumption), 
        \begin{align*}
            \mu^*(E) &= \mu^*(E \cap A) + \mu^*(E \cap A^c)\\ 
            &= \mu^*(E \cap A \cap B) + \mu^*(E \cap A \cap B^c) + \mu^*(E \cap A^c \cap B) + \mu^*(E \cap A^c \cap B^c)\\
        \end{align*}
        
        Now consider $A \cup B$. Using set algebra, 
        \[A \cup B = (A \cap B) \cup (A \cap B^c) \cup (B \cap A^c)\]
        (and this matches the first three terms above very nicely)

        Then 
        \[E \cap (A \cup B) = (E \cap A \cap B) \cup (E \cap A \cap B^c) \cup (E \cap B \cap A^c)\]
        
        By subadditivity of the outer measure, 
        \[\mu^*(E \cap (A \cup B)) \leq \mu^*(E \cap A \cap B) + \mu^*(E \cap A \cap B^c) + \mu^*(E \cap B \cap A^c)\]

        Further, 
        \begin{align*}
            \mu^*(E \cap A \cap B) &+ \mu^*(E \cap A \cap B^c) + \mu^*(E \cap A^c \cap B) + \mu^*(E \cap A^c \cap B^c)\\ 
                &\geq mu^*(E \cap A \cap B) + \mu^*(E \cap A \cap B^c) + \mu^*(E \cap B \cap A^c)
        \end{align*}
        which is just 
        \[\mu^*(E) \geq \mu^*(E \cap (A\cup B)) + \mu^*(E \cap (A \cup B)^c)\]

        Now take $\mu^*(A \cup B)$. Using the above, 
        \[\mu^*(A \cup B) = \mu^*(A \cup B \cap A) + \mu^*(A \cup B \cap A^c) = \mu^*(A) + \mu^*(B)\]
        which is what we wanted to show. 

        Now we can inductively extend this pairwise additivity to a finite union.

        Let $A_i \in \M$ and 
        \[\bigcup_{i=1}^N A_i \in \M\]
        with $A_i \cap A_j = \emptyset$ for $i \neq j$

        By induction, 
        \[\mu^*\left(\bigcup_{i=1}^N A_i\right) = \sum_{i=1}^{N} \mu^*(A_i)\]
        
        \div 

        STEP 3 (Countable Additivity): Let $A_i \in \M$ and $\bigcup_{i=1}^\infty A_i \in \M$ with $A_i \cap A_j = \emptyset$ for $i \neq j$.

        Define 
        \[B_n = \bigcup_{j=1}^n A_j\]

        Take a test set $E$ with $\mu^*(E) < \infty$. 

        By induction on the Carathéodory Criterion,
        \begin{align*}
            \mu^*(E) &= \mu^*(E \cap B_j) + \mu^*(E \cap B_j^c)\\ 
                &= \mu^*\left(E \cap \bigcup_{j=1}^n A_j\right) + \mu^*\left(E \cap \bigcap_{j=1}^n A_j^c\right)\\ 
                &= \sum_{j=1}^{n} \mu^*(E \cap A_j) + \mu^*(E \cap \bigcap_{j=1}^n A_j^c) \qquad \text{Step 2}\\ 
                &\geq \sum_{j=1}^{n} \mu^*(E \cap A_j) + \mu^*(E \cap \bigcap_{j=1}^{\infty} A_j^c)\\ 
                &\geq \sum_{j=1}^{\infty} \mu^*(E \cap A_j) + \mu^*(E \cap \bigcap_{j=1}^n A_j^c)\\ 
                &\geq \mu^*\left(E \cap \bigcup_{i=1}^\infty  A_i\right) + \mu^*\left(E \cap \bigcap_{i=1}^{\infty} A_j^c\right) \qquad \text{Subadditivity}
        \end{align*}

        \div

        STEP 4 (Completeness) Let $\mu^*(A) = 0$, then $A \in \M$. 

        Take any $E \sub P(X)$. We want to show 
        \[\mu^*(E) \geq \mu^*(E \cap A) + \mu^*(E \cap A^c)\]

        But by monotonicity, 
        \[\mu^*(E \cap A) \leq \mu^*(A) = 0 \implies \mu^*(E \cap A) = 0\]

        Using monotonicity again, we get the inequality. Hence, every set with outer-measure $0$ is in $\M$.  

        Now take $A_1 \sub A$. By monotonicity, $\mu^*(A_1) = 0 \in \M$    
    \end{tbox}

    \textbf{Completeness:} A measure space $(X, \M, \mu)$ is complete if $\forall A \in \M$ with $\mu(A) = 0$, then $B \in \M$ for all $B \sub A$.

    \begin{exercise}
        \textbf{Exercise:} Use the Carathéodory Procedure to produce the Hausdorff Measure (HW 4)
    \end{exercise}

\subsection*{Lebesgue Measure} 
    On the real numbers, it would be very nice to have a measure $\mu$ such that $\mu((a, b)) = \rho(a, b) = b - a$.
    
    \textbf{Lemma:} If $A \sub \bigcup_{i=1}^\infty (a_i, b_i)$,
    \[\mu^*(A) = \inf \sum_{n=1}^{\infty} \rho(a_i, b_i)\] 
    and $\mu^*((a, b)) = b - a$. 
   
    Then using the Carathéodory process, we get the Lebesgue Measure on $(\R, \M, \mu)$

    \begin{tbox}{\textbf{Proposition (Faithfulness of the Lebesgue Measure):} Let $I$ be any interval (closed, open, half-open, etc.) on $\R$. Then $\mu(I) = \rho(I)$}
        \emph{Proof:} 
        
        STEP 1. Suppose $I = [a, b]$ is closed and finite. $\forall \ep > 0$, consider $(a - \ep, b + \ep) \supset [a, b]$.

        By definition of $\inf$, 
        \[\mu^*([a, b]) \leq \rho((a - \ep, b + \ep)) = b - a + 2\ep\]

        But by arbitrariness of $\ep$, $\mu^*([a, b]) \leq b - a$

        On the other hand, take $\bigcup_{i=1}^\infty (a_i, b_i) \supseteq [a, b]$. By Heine-Borel, there exists a finite cover for $[a, b]$ so we can take 
        \[[a, b] \sub \bigcup_{i=1}^N (a_i, b_i)\]

        We want to show that 
        \[\sum_{i=1}^{N} (b_i - a_i) \geq b - a \implies \mu^*([a, b]) \geq b - a\]

        $a$ must be in some open interval, so call it $a \in (a_1, b_1)$. WLOG, suppose $b_1 \leq b$. 
    
        But then $b_1 \in (a_2, b_2)$. If $b_2 > b$, then $(a_1, b_1) \cup (a_2, b_2)$ would cover $[a, b]$ and 
        \[b_2 - a_2 + b_1 - a_1 \geq b - a\]

        Therefore, assume $b_2 \leq b$. Inductively define $b_n \in (a_{n+1}, b_{n+1})$. But because we have a finite cover, this process is not infinite, i.e. for some $N$, $b_N > b$. 

        Then it suffices to show 
        \[b_N - a_N + b_{N-1} - a_{N-1} + \dots + b_1 - a_1 \geq b - a\]
        and by construction, each $-a_{i} + b_{i-1} > 0$

        \div

        STEP 2. Now take any interval $I$. Consider
        \[[a + \ep, b - \ep] \sub I \sub (a - \ep, b + \ep)\] 

        But this is an open cover so 
        \[\mu^*(I) \leq b - a + 2\ep\] 

        But in part 1, we showed that 
        \[b - a - 2\ep \leq \mu^*(I)\]

        So by arbitrariness of $\ep$, $\mu^*(I) = b - a$
    \end{tbox}

\section{Oct 03}
    \begin{tbox}{\textbf{Corollary:} If $A$ is a countable subset of $\R$, $\mu^*(A) = 0$. Furthermore, $[0, 1]$ is not countable. }
        \emph{Proof:} $\forall x \in \R$, $\{x\} = (x - \ep, x + \ep)$, so 
        \[\mu^*(\{x\}) \leq 2\ep \implies \mu^*(\{x\})= 0\]

        Now suppose $A = \bigcup_{n=1}^\infty a_n$ with $a_n \in \R$. 

        By subadditivity, 
        \[\mu^*(A) \leq \sum_{n=1}^{\infty} \mu^*(\{a_n\}) = 0\]
        
        \div 

        Since $\mu^*([0, 1]) = 1 \neq 0$, $[0, 1]$ is not countable.
    \end{tbox}

    \begin{tbox}{\textbf{Proposition:} $\B_{\R} \sub \M$}
        \emph{Proof:} By the characterization of the Borel set on $\R$, it suffices to show that $(a, \infty) \in \M$ for all $a \in \R$. 

        $\forall E \in \M$ with $\mu^*(E) < \infty$, we want to show 
        \begin{align*}
            \mu(E) &\geq \mu^*(E \cap (a, \infty)) + \mu^*(E \cap (a, \infty)^c)\\ 
            &= \mu^*(E \cap (a, \infty)) + \mu^*(E \cap (-\infty, a])\\
        \end{align*}

        For notational convenience, let 
        \begin{align*}
            E_1 &= E \cap (a, \infty)\\ 
            E_2 &= E \cap (-\infty, a]
        \end{align*}

        Let $\ep > 0$. By the sharpness of the outer measure, $\exists \bigcup_{n=1}^\infty I_n$ such that  
        \[E \sub \bigcup_{n=1}^\infty I_n\]
        and 
        \[\sum_{n=1}^{\infty} \abs{I_n} < \mu^*(E) + \ep\]

        Then 
        \begin{align*}
            E_1 &\sub \bigcup_{n=1}^\infty I_n \cap (a, \infty)\\ 
            E_2 &\sub \bigcup_{n=1}^\infty I_n \cap (-\infty, a]
        \end{align*}
        so 
        \begin{align*}
            \mu^*(E_1) &\leq \sum_{n=1}^{\infty} \mu^*(I_n \cap (a, \infty))\\ 
            \mu^*(E_2) &\leq \sum_{n=1}^{\infty} \mu^*(I_n \cap (-\infty, a])
        \end{align*}

        Now by the faithfulness of the Lebesgue Measure,
        \[\mu(I_n) = \mu^*(I_n \cap (a, \infty)) +  \mu^*(I_n \cap (-\infty, a])\]
        so 
        \[\mu^*(E_1) + \mu^*(E_2) \leq \sum_{n=1}^{\infty} \mu(I_n) \leq \mu^*(E)\]
    \end{tbox}

\subsection*{Transformations}

\textbf{Definitions:} Given $E \sub \R$, we define 
\begin{itemize}
    \item (Translation) $E + a := \{x + a: x \in E\}$
    \item (Dilation) $rE := \{rx: x \in E\}$
\end{itemize}

\begin{tbox}{\textbf{Lemma:} For the Lebesgue outer measure and $E \in \M$, 
    \begin{enumerate}
        \item $\mu^*(E + a) = \mu^*(E)$
        \item $\mu^*(rE) = \abs{r}\mu^*(E)$
    \end{enumerate}}
    \emph{Proof Sketch:} Let $E \sub \bigcup_{n=1}^\infty I_n$. 

    Certainly, 
    \begin{align*}
        E + a &\sub \bigcup_{n=1}^\infty \{I_n + a\}\\ 
        rE &\sub \bigcup_{n=1}^\infty \{\abs{r}I_n\}
    \end{align*}

    Then
    \[\sum_{n=1}^{\infty} \rho(I_n) = \sum_{n=1}^\infty \rho(I_n+ a) \geq \mu^*(E + a)\]

    And taking the infimum,
    \[\mu(E) \geq \mu^*(E + a)\]

    For dilation, notice $\abs{I_n} = \frac{1}{\abs{r}} \abs{rI_n}$. The result follows similarly. 

    The other direction is exactly the same. 
\end{tbox}

\subsection*{Approximation of Measurable Sets}
    \begin{tbox}{\textbf{Lemma:} 
        \begin{enumerate}
            \item (Approximation from Above) $\forall E \sub P(X)$ and $\forall \ep > 0$, then exists an open set $O$ such that $E \sub O$ and 
            \[\mu(O) \geq \mu^*(E) \geq \mu(O) - \ep\]
            \item (Approximation from Below) $\forall E \sub \M$ and $\forall \ep > 0$, $\exists K$ closed such that 
            \[\mu(K) \leq \mu(E) \leq \mu(K) + \ep\] 
        \end{enumerate}}
        \emph{Proof:} 

        1. By definition of $\mu^*(E)$, $\exists O = \bigcup_{n=1}^\infty I_n \supset E$ such that 
        \[\mu^*(E) \geq \sum_{n=1}^{\infty} \rho(I_n) - \ep\] 
        and by subadditivity, $\mu^*(E) \geq \mu^*(O) - \ep$. 
        
        2. Assume $E \sub [a, b]$. Consider $E^c \cap [a, b]$. By part 1, $\exists O \supset E^c \cap [a, b]$ such that 
        \[\mu^*(E^c \cap [a, b]) \geq \mu^*(O) - \ep\]
        and with some algebra, 
        \[\abs{b - a} - \mu^*(E^c) \leq \abs{b - a} - \mu^*(O) + \ep\]

        By measurability, 
        \begin{align*}
            \mu(E) &\leq \abs{b - a} - \mu^*(O \cap [a, b]) - \mu^*(O \cap [a, b]^c) + \ep\\ 
            &\leq \abs{b - a} - \mu^*(O \cap [a, b])\\ 
            &= \mu^*([a, b] \cap O^c) + \ep
        \end{align*}

        \begin{exercise}
            \textbf{Exercise:} Complete the case $E \not\sub [a, b]$
        \end{exercise}
    \end{tbox}

\chapter{Measurable functions}
\section{Oct 03}
    \textbf{Measurable function:} Let $(X, \M, \mu)$ be a measure space. Let $f: X \to \R$. $f$ is \emph{measurable} iff $\forall \alpha \in \R$, 
    \[\{x \in X, f(x) > \alpha\} \in \M\] 

    Equivalently, $f$ is measurable iff $\forall \alpha \in \R$, $\{f \geq \alpha\}$ is measurable. 

    \begin{tbox}{\textbf{Proposition:} If $f, g$ are measurable, so is
            \begin{enumerate}[label=(\alph*)]
                \item $f + c$
                \item $f + g$
                \item $cf$
                \item $fg$
            \end{enumerate} 
            for $c \in \R \setminus \{0\}$}
        \emph{Proof:}
        1a. For all $\alpha \in \R$,
        \[\{x: f(x) + c > \alpha\} = \{x: f(x) > \alpha - c\} \in \M\] `
        
        1b. Consider $\{x: f(x) + g(x) > \alpha\}$. We claim 
        \[\{x: f(x) + g(x) > \alpha\} = \bigcup_{r \in \Q} \{f(x) > r\} \cap \{g(x) > x - r\}\]

        \begin{proof}
            \emph{Proof:} Certainly, 
            \[\{x: f(x) + c > \alpha\} = \{x: g(x) > \alpha - f(x)\}\] 
            and for fixed $x$, we can invoke the density and countability of the rational numbers...
        \end{proof}
    \end{tbox}

    \begin{tbox}{\textbf{Proposition:} If $\{f_n\}$ are measurable, so are 
        \begin{enumerate}[label=(\alph*)]
            \item $\sup_n f_n$
            \item $\inf_n f_n$
            \item $\limsup_n f_n$
            \item $\liminf_n f_n$
        \end{enumerate}}
        \emph{Proof:} ...
    \end{tbox}
\end{document}  