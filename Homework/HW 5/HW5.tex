\documentclass[12pt]{article} 
\usepackage[utf8]{inputenc}
\usepackage{geometry}
\geometry{letterpaper, margin=0.5in}
\usepackage{graphicx} 
\usepackage{bbm}
\usepackage{parskip}
\usepackage{booktabs}
\usepackage{array} 
\usepackage{paralist} 
\usepackage{verbatim}
\usepackage{subfig}
\usepackage{fancyhdr}
\usepackage{sectsty}
\usepackage[shortlabels]{enumitem}

\pagestyle{fancy}
\renewcommand{\headrulewidth}{0pt} 
\lhead{}\chead{}\rhead{}
\lfoot{}\cfoot{\thepage}\rfoot{}


%%% ToC (table of contents) APPEARANCE
\usepackage[nottoc,notlof,notlot]{tocbibind} 
\usepackage[titles,subfigure]{tocloft}
\renewcommand{\cftsecfont}{\rmfamily\mdseries\upshape}
\renewcommand{\cftsecpagefont}{\rmfamily\mdseries\upshape} %

\usepackage{amsmath}
\usepackage{amssymb}
\usepackage{mathtools}
\usepackage{empheq}
\usepackage{xcolor}

\usepackage{tikz}
\usepackage{pgfplots}
\usepackage{tikz-cd}
\pgfplotsset{compat=1.18}
\newcommand{\ans}[1]{\boxed{\text{#1}}}
\newcommand{\vecs}[1]{\langle #1\rangle}
\renewcommand{\hat}[1]{\widehat{#1}}

\renewcommand{\P}{\mathbb{P}}
\newcommand{\R}{\mathbb{R}}
\newcommand{\E}{\mathbb{E}}
\newcommand{\Z}{\mathbb{Z}}
\newcommand{\N}{\mathbb{N}}
\newcommand{\Q}{\mathbb{Q}}
\newcommand{\C}{\mathbb{C}}

\newcommand{\ind}{\mathbbm{1}}
\newcommand{\qed}{\quad \blacksquare}

\newcommand{\brak}[1]{\left\langle #1 \right\rangle}
\newcommand{\bra}[1]{\left\langle #1 \right\vert}
\newcommand{\ket}[1]{\left\vert #1 \right\rangle}

\newcommand{\abs}[1]{\left\vert #1 \right\vert}
\newcommand{\mfX}{\mathfrak{X}}
\newcommand{\ep}{\varepsilon}

\newcommand{\Ec}{\mathcal{E}}
\newcommand{\A}{\mathcal{A}}
\newcommand{\Fc}{\mathcal{F}}
\newcommand{\Cc}{\mathcal{C}}
\newcommand{\B}{\mathcal{B}}
\newcommand{\M}{\mathcal{M}}
\newcommand{\X}{\chi}

\newcommand{\sub}{\subseteq}
\newcommand{\st}{\text{ s.t. }}
\newcommand{\card}{\text{card }}
\renewcommand{\div}{\vspace*{10pt}\hrule\vspace*{10pt}}
\newcommand{\surj}{\twoheadrightarrow}
\newcommand{\inj}{\hookrightarrow}
\newcommand{\biject}{\hookrightarrow \hspace{-8pt} \rightarrow}
\renewcommand{\bar}[1]{\overline{#1}}
\newcommand{\overcirc}[1]{\overset{\circ}{#1}}
\newcommand{\diam}{\text{diam }}

\newcommand*{\tbf}[1]{\ifmmode\mathbf{#1}\else\textbf{#1}\fi}

\usepackage{tcolorbox}
\tcbuselibrary{breakable, skins}
\tcbset{enhanced}
\newenvironment*{tbox}[2][gray]{
    \begin{tcolorbox}[
parbox=false,
        colback=#1!5!white,
        colframe=#1!75!black,
        breakable,
        title={#2}
    ]}
    {\end{tcolorbox}}

\newenvironment*{proof}[1][blue]{
    \begin{tcolorbox}[
    parbox=false,
        colback=#1!5!white,
        colframe=#1!75!black,
        coltext=#1,
        breakable
    ]}
    {\end{tcolorbox}}

\title{APMA 2110: HW 5}
\author{Milan Capoor}
\date{10/21/24}

\begin{document}
\maketitle
1. Show that the characteristic function $\X_E = \ind_E$ is a measurable function iff $E$ is a measurable set.

\color{blue}
    $\implies$. Assume $\ind_E$ is a measurable function. 

    By definition, $\forall \alpha \in \R$, $\{x \in X: \ind_E(x) > \alpha\}$ is measurable.

    WLOG, let $\alpha = 0$. Then 
    \[\{x \in X: \ind_E(x) > \alpha\} = \{x \in X: \ind_E(x) > 0\}\]

    But since $\ind_E: X \to \{0, 1\}$,
    \[\{x \in X: \ind_E(x) > 0\} = \{x \in X: \ind_E(x) = 1\} = \{x \in X: x \in E\} = E\]

    Hence, $E$ is measurable. 

    \div

    $\impliedby$. Now assume $E$ is measurable. 

    We want to show that $A = \{x \in X: \ind_E(x) > \alpha\}$ is measurable for all $\alpha \in \R$. 

    CASE 1. If $\alpha > 1$, then $A = \emptyset \in \M$. 

    CASE 2. If $0 \leq \alpha < 1$, then 
    \begin{align*}
        A &= \{x \in X: \ind_E(x) > \alpha\} \\
        &= \{x \in X: \ind_E(x) = 1\} \\
        &= \{x \in X: x \in E\} \\
        &= E
    \end{align*}
    so $A$ is measurable. 

    CASE 3. If $\alpha <0$, then $A = X \in \M$. $\qed$
    
\color{black}



\pagebreak

2. Let $\{f_n\}$ be a sequence of measurable functions on $X$ then $\{x : \exists\lim f_n(x)\}$ is a measurable set.

    \color{blue}
        \tbf{Lemma:} $\exists \lim f_n(x) \iff \limsup f_n(x) = \liminf f_n(x) = \lim f_n(x)$. 

        \begin{proof}
            \emph{Proof:} ($\implies$) Let $\ep > 0$. If $\lim f_n(x) = f(x)$, then $\exists N \in \N$ such that $\rho(f_n(x), f(x)) < \ep$ for all $n \geq N$. 
            
            Then for $n \geq N$, $\{f_k(x): k \geq n\} \in B(f_n(x), \ep)$ so 
            \[\limsup f_n = \lim_{n \to \infty} (\sup \{f_k(x): k \geq n\}) \in B(f_n(x), \ep)\]
            and 
            \[\liminf f_n = \lim_{n \to \infty} (\inf \{f_k(x): k \geq n\}) \in B(f_n(x), \ep)\]

            Since $\ep$ arbitrary, $\limsup f_n(x) = \liminf f_n(x) = f(x)$.
        
            ($\impliedby$) Let $\ep > 0$ and denote $f(x) = \lim f_n(x)$. Since $\limsup f_n(x) = \liminf f_n(x) = f(x)$, $\exists N \in \N$ such that for $n \geq N$, 
            \begin{align*}
                \rho(\limsup f_n(x), f(x)) &= \rho(\sup_{k \geq n} f_k(x), f(x)) < \ep \implies \sup_{k \geq n} f_k(x) \in B(f(x), \ep)\\ 
                \rho(\liminf f_n(x), f(x)) &= \rho(\inf_{k \geq n} f_k(x), f(x)) < \ep \implies \inf_{k \geq n} f_k(x) \in B(f(x), \ep)
            \end{align*}

            But 
            \[\inf_{k \geq n} f_k(x) \leq f_n(x) \leq \sup_{k \geq n} f_k(x)\]
            by the definitions of $\inf$ and $\sup$ so 
            \[\rho(f_n(x), f(x)) = \max\left(\rho(\inf_{k \geq n} f_k(x), f(x)), \rho(\sup_{k \geq n} f_k(x), f(x))\right) = \max(\ep, \ep) = \ep\]
            so for $n$ sufficiently large, $\rho(f_n(x), f(x)) < \ep$. Hence, $\lim f_n(x) = f(x)$.
        \end{proof} 

        Call $f(x) = \limsup f_n(x)$ and $g(x) = \liminf f_n(x)$. By propositions from class, $f$, $g$, and $f - g$ are measurable functions because $\{f_n\}$ are measurable. 
        
        Therefore, by the Lemma,
        \begin{align*}
            \{x: \exists \lim f_n(x)\} &= \{x: f(x) = g(x)\} \\
            &= \{x: f(x) - g(x) = 0\}
        \end{align*}

        Since $f - g$ is measurable, $\{x: f(x) - g(x) > \alpha\}$ is measurable for all $\alpha \in \R$. 
        
        Let $\ep > 0$ so $\{x: f(x) - g(x) > \ep\} \in \M$ and $\{x: f(x) - g(x) > -\ep\} \in \M$. 

        Then since $\M$ is closed under complements and countable intersections, 
        \[\{x: f(x) - g(x) > \ep\}^c = \{x: f(x) - g(x) \leq \ep\} \in \M\]
        and 
        \[\{x : f(x) - g(x) = 0\} = \{x : f(x) - g(x) \leq \ep\} \cap \{x: f(x) - g(x) > -\ep\} \in M \qed\]
        
    \color{black}


\pagebreak 

3. Let $E$ be a Lebesgue measurable set in $\R$ and $\mu(E) > 0$. Show that for any $\alpha <1$, there exists an open interval $I_{\alpha}$ such that $\mu(E \cap I_{\alpha}) > \alpha \mu(I_{\alpha})$.

    \color{blue}
        Suppose not. Then for all open intervals $I$, $\mu(E \cap I) \leq \alpha \mu(I)$.

        Let $\ep > 0$. By approximation from above, $\exists O$ open such that $E \sub O$ and 
        \[\mu(O) - \ep \leq \mu(E) \leq \mu(O)\]

        But since $O \sub \R$, by a proposition from class, we can write $O$ as a countable union of disjoint open intervals, $O = \bigcup_{n} I_{n}$.

        Then,
        \begin{align*}
            \mu(E) &= \mu(\bigcup_{n} E \cap I_{n}) \qquad (E \sub O) \\ 
            &= \sum_{n} \mu(E \cap I_{n}) \qquad (I_{n} \text{ disjoint})\\ 
            &\leq \sum_{n} \alpha \mu(I_{n}) \qquad (\text{by assumption})\\
            &= \alpha \sum_n \mu(I_n)\\ 
            &= \alpha \mu(O) \qquad (\text{disjoint union})
        \end{align*}

        So 
        \[\mu(O) - \ep \leq \mu(E) = \alpha \mu(O) \leq \mu(O)\]

        Taking $\ep \to 0$, we have $\mu(O) = \alpha \mu(O)$ but $\alpha \neq 1$ and $\mu(E) > 0 \implies \mu(O) > 0$ by monotonicity, so we have a contradiction. 
    \color{black}



\end{document}