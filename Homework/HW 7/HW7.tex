\documentclass[12pt]{report} 
\usepackage[utf8]{inputenc}
\usepackage{geometry}
\geometry{letterpaper,
    margin=0.5in}
\usepackage{graphicx} 
\usepackage{parskip}
\usepackage{booktabs}
\usepackage{array} 
\usepackage{paralist} 
\usepackage{verbatim}
\usepackage{subfig}
\usepackage{fancyhdr}
\usepackage{sectsty}
\usepackage[shortlabels]{enumitem}

\pagestyle{fancy}
\renewcommand{\headrulewidth}{0pt} 
\lhead{}\chead{}\rhead{}
\lfoot{}\cfoot{\thepage}\rfoot{}

%%% ToC (table of contents) APPEARANCE
\usepackage[nottoc,notlof,notlot]{tocbibind} 
\usepackage[titles,subfigure]{tocloft}
\renewcommand{\cftsecfont}{\rmfamily\mdseries\upshape}
\renewcommand{\cftsecpagefont}{\rmfamily\mdseries\upshape} %

\usepackage{amsmath}
\usepackage{amssymb}
\usepackage{mathtools}
\usepackage{empheq}
\usepackage{xcolor}
\usepackage{bbm}
\usepackage{tikz}
\usepackage{pgfplots}
\usepackage{tikz-cd}
\pgfplotsset{compat=1.18}

\newcommand{\ans}[1]{\boxed{\text{#1}}}
\newcommand{\vecs}[1]{\langle #1\rangle}
\renewcommand{\hat}[1]{\widehat{#1}}

\renewcommand{\P}{\mathbb{P}}
\newcommand{\R}{\mathbb{R}}
\newcommand{\E}{\mathbb{E}}
\newcommand{\Z}{\mathbb{Z}}
\newcommand{\N}{\mathbb{N}}
\newcommand{\Q}{\mathbb{Q}}
\newcommand{\C}{\mathbb{C}}

\newcommand{\ind}{\mathbbm{1}}
\newcommand{\qed}{\quad \blacksquare}

\newcommand{\brak}[1]{\left\langle #1 \right\rangle}
\newcommand{\bra}[1]{\left\langle #1 \right\vert}
\newcommand{\ket}[1]{\left\vert #1 \right\rangle}

\newcommand{\abs}[1]{\left\vert #1 \right\vert}
\newcommand{\mfX}{\mathfrak{X}}
\newcommand{\ep}{\varepsilon}

\newcommand{\Ec}{\mathcal{E}}
\newcommand{\A}{\mathcal{A}}
\newcommand{\F}{\mathcal{F}}
\newcommand{\Cc}{\mathcal{C}}
\newcommand{\B}{\mathcal{B}}
\newcommand{\M}{\mathcal{M}}
\newcommand{\X}{\chi}
\renewcommand{\L}{\mathcal{L}}

\newcommand{\sub}{\subseteq}
\newcommand{\st}{\text{ s.t. }}
\newcommand{\card}{\text{card }}
\renewcommand{\div}{\vspace*{10pt}\hrule\vspace*{10pt}}
\newcommand{\surj}{\twoheadrightarrow}
\newcommand{\inj}{\hookrightarrow}
\newcommand{\biject}{\hookrightarrow \hspace{-8pt} \rightarrow}
\renewcommand{\bar}[1]{\overline{#1}}
\newcommand{\overcirc}[1]{\overset{\circ}{#1}}
\newcommand{\diam}{\text{diam }}

\renewcommand{\Re}{\text{Re}\,}
\renewcommand{\Im}{\text{Im}\,}
\newcommand{\sign}{\text{sign}\,}

\newcommand*{\tbf}[1]{\ifmmode\mathbf{#1}\else\textbf{#1}\fi}

\usepackage{tcolorbox}
\tcbuselibrary{breakable, skins}
\tcbset{enhanced}
\newenvironment*{proof}[1][blue]{
\begin{tcolorbox}[
    parbox=false,
    colback=#1!5!white,
    colframe=#1!75!black,
    breakable
]}
{\end{tcolorbox}}

\title{APMA 2110: Homework 7}
\author{Milan Capoor}
\date{4 Nov 2024}

\begin{document}
\maketitle
1. Let $f \geq 0$ be a measurable function and define its distribution function for $\lambda \geq 0$,
\[d_f(\lambda) = \mu\{x: \abs{f(x)} > \lambda\}\]

Show 
\[\int \abs f \; d\mu = \int_0^{\infty} d_f(\lambda) \; d\lambda\]

    \color{blue}
        First, $f \geq 0$ so $\abs{f} = f$. 

        Since $f$ is measurable, we can take a sequence of simple functions $\phi_n \to f$ such that $0 \leq \abs{\phi_n} \leq \abs{\phi_{n+1}} \leq \abs{f}$ for $n \geq 1$. 

        By the Monotone Convergence Theorem,
        \[\int f \; d\mu = \int \lim phi_n \; d\mu = \lim \int \phi_n \; d\mu\]

        Each $\phi_n$ is a simple function, i.e. $\phi_n = \sum_{i=1}^{m_n} a_i^{(n)} \ind_{A_i^{(n)}}(x)$, so
        \begin{align*}
            \int \phi_n \; d\mu &= \sum_{i=1}^{m_n} a_i^{(n)} \mu(A_i^{(n)})\\ 
                &= \sum_{i=1}^{m_n} (a_i^{(n)} - a_{i-1}^{(n)})\; \mu(x: \phi_n(x) > a_{i-1}^{(n)})
        \end{align*}

        By definition,
        \begin{align*}
            \int f \; d\mu &= \sup_{\phi \leq f} \int \phi \; d\mu\\ 
            &= \lim_{n \to \infty} \int \phi_n \; d\mu && (\text{monotonicity of } \phi_n)\\ 
            &= \lim_{n \to \infty} \sum_{i=1}^{m_n} (a_i^{(n)} - a_{i-1}^{(n)})\; \mu(x: \phi_n(x) > a_{i-1}^{(n)})
        \end{align*}

        But since $\phi_n \to f$, $\forall \ep > 0, \exists N \in \N \st \forall n \geq N, \; \abs{f - \phi_n} < \ep$. WLOG, consider 
        \[\phi := \phi_N = \sum_{i=1}^m a_i \ind_{A_i}\]
        so 
        \begin{align*}
            \int \phi \; d\mu &= \sum_{i=1}^{\infty} (a_i - a_{i-1})\mu\{x: \phi(x) > a_{i-1}\}\\ 
            &= \sum_{i=1}^{\infty} (a_i - a_{i-1})\left[\mu\{x: f(x) > a_{i-1}\} - \mu\{a_{i-1}: a_{i-1} < \abs{f - \phi}\} \right]\\ 
            &= \sum_{i=1}^{\infty} (a_i - a_{i-1})\left[\mu\{x: f(x) > a_{i-1}\} - \mu\{a_{i-1}: a_{i-1} < \ep\} \right]\\ 
            &= \sum_{i=1}^{\infty}(a_i - a_{i-1})\mu\{x: f(x) > a_{i-1}\}\\ 
            &= \int_0^{\infty} \mu\{x: f(x) > \lambda\} \; d\lambda \qed
        \end{align*}

    \color{black}


\pagebreak 

2. Let $1 < a \in \R$. Determine the limit of the following Lebesgue integral:
\[\lim_{n \to \infty} \int_0^{\infty} (1 + \frac{t}{n})^n e^{at}\; dt\]

    \color{blue}
        Let $f_n(t) = (1 + \frac{t}{n})^n e^{-at}$. Notice that $f_{n+1}(t) \geq f_{n}(t)$ for all $t \geq 0$

        \begin{proof}
            \emph{Proof:} It suffices to show that 
            \[(1 + \frac{t}{n})^n \leq (1 + \frac{t}{n+1})^{n+1}\]

            Consider  
            \begin{align}
                (1 + \frac{t}{n+1})^{n+1} - (1 + \frac{t}{n})^{n} &= (1 + \frac{t}{n+1})(1 + \frac{t}{n+1})^n - (1 + \frac{t}{n})^{n}\\ 
                &\geq (1 + \frac{t}{n+1})(1 + \frac{t}{n+1})^n - (1 + \frac{t}{n+1})^{n}\\ 
                &= (1 + \frac{t}{n+1})^n[(1 + \frac{t}{n+1}) - 1]\\
                &= (1 + \frac{t}{n+1})^n[\frac{t}{n+1}]\\
                &\geq 1 \cdot 0 = 0
            \end{align}
        \end{proof}

        So, by the Monotone Convergence Theorem,
        \begin{align*}
            \lim_{n \to \infty} \int_0^{\infty} (1 + \frac{t}{n})^n e^{-at}\; dt &= \int_0^{\infty} \lim_{n \to \infty} (1 + \frac{t}{n})^n e^{-at}\; dt \\ 
            &= \int_0^{\infty} e^t e^{-at} \; dt && (\text{by definition of exp})\\
            &= \int_0^{\infty} e^{-(a-1)t} \; dt\\
            &= \left[\frac{e^{-(a-1)t}}{1 - a}\right]_0^{\infty}\\ 
            &= \frac{0}{1-a} - \frac{1}{1-a} = \boxed{\frac{1}{a-1}}
        \end{align*}
    \color{black}
\pagebreak 

3. 
\begin{enumerate}[(a)]
    \item Let $\{r_n\}_1^{\infty}$ be an enumeration of $\Q \cap [0, 1]$. Let $0 < a_n < \infty$ with $\sum_{n=1}^{\infty} a_n < \infty$.

    Prove that the series 
    \[h(t) = \sum_{n=1}^\infty \frac{a_n}{\sqrt{\abs{t - r_n}}}\]
    converges a.e. (WRT the Lebesgue measure) for $t \in [0,1]$ 

        \color{blue}
            We want to show that $\mu\{t: h(t) = \infty\} = 0$.

            CASE 1. $t \in \Q \cap [0, 1]$. Then, $\exists r_n$ such that $t = r_n$ so $\sqrt{\abs{t - r_n}} = 0$ and $\frac{a_n}{\sqrt{\abs{t - r_n}}} = \infty \implies h(t) = \infty$. 

            However, $Q \cap [0, 1]$ is countable so $\mu\{t: t \in \Q \cap [0, 1] \land h(t) = \infty\} = 0$.

            \div

            CASE 2. $t \in [0, 1] \setminus \Q$. Let $\ep > 0$.

            Define 
            \[E_n(\ep) = \left\{t \in [0, 1] \setminus \Q \bigg\vert \frac{a_n}{\sqrt{\abs{t - r_n}}} > \ep\right\}\]

            We WTS that $\mu(E_n(\ep)) = 0$.  

            Certainly, if $t \in E_n(\ep)$, then $\abs{t - r_n} < (\frac{a_n}{\ep})^2$. By the faithfulness of the Lebesgue measure, $\mu(E_n(\ep)) \leq 2(\frac{a_n}{\ep})^2$. 

            We know that $\lim_{\ep \to \infty} 2(\frac{a_n}{\ep})^2 = 0$ so $\frac{a_n}{\sqrt{\abs{t - r_n}}} < \infty$ a.e. 

            It only remains to show that $\sum_{n=1}^{\infty} \frac{a_n}{\sqrt{\abs{t - r_n}}} < \infty$ a.e.

            Let 
            \[E(\ep) = \{t \in [0, 1] \setminus \Q \; | \; h(t) > \ep\}\]

            Then, $E(\ep) \sub \bigcup_n E_n(\ep)$, so 
            \begin{align*}
                \mu(E(\ep)) &\leq \mu\left(\bigcup_n E_n(\ep)\right)\\ 
                &\leq \sum_n \mu(E_n(\ep))\\
                &\leq \sum_n 2(\frac{a_n}{\ep})^2\\
                &= 2\ep^{-2} \sum_n a_n^2
            \end{align*}

            \tbf{Lemma:} $\sum_n a_n^2 \leq \left(\sum_n a_n\right)^2$

            \begin{proof}
                \emph{Proof:} 
                \[\left(\sum_n a_n\right)^2 = \sum_n \sum_m a_n a_m = \sum_n a_n^2 + \sum_n \sum_{n\neq m} a_n a_m \geq \sum_n a_n^2\]
            \end{proof}

            So $\mu(E(\ep)) \leq 2\ep^{-2} \left(\sum_n a_n\right)^2 < \infty$ for all fixed $\ep >0$. And indeed, 
            \[\lim_{\ep\to \infty} \mu\{t \in [0, 1] \setminus \Q \; | \; h(t) > \ep\} =\mu(E(\ep)) = 0\]
            so $h(t) < \infty$ a.e.
        \color{black}

    \item If $g = h$ a.e., prove that $g$ is unbounded in every subinterval of $[0, 1]$.
    
        \color{blue}
            Suppose $g = h$ a.e. and let $I \sub [0, 1]$ be a subinterval. 

            Suppose $g$ is bounded on $I$. Then, $\exists M \in \R \st \forall t \in I, \; g(t) \leq M$. 

            Once again, we proceed by cases. 

            CASE 1. $\{t \in I \cap \Q: h(t) = g(t)\} \neq \emptyset$. Then, we can take $t = r_n \in I \cap \Q$ such that $h(t) = g(t) = \infty$ which is a contradiction.

            \div 

            CASE 2. $\{t \in I \setminus \Q: h(t) = g(t)\} \neq \emptyset$. 

            By the density of $\Q$ in $\R$, $\exists r_n \in I \cap \Q$ such that $\abs{t - r_n} < \delta$ for some $\delta > 0$. 

            But then, for all $t \in I \setminus \Q$, $\frac{a_n}{\sqrt{\abs{t - r_n}}} > \frac{a_n}{\sqrt \delta}$ so by taking $\delta \to 0$, we have $\frac{a_n}{\sqrt{\abs{t - r_n}}} \to \infty$. But this implies that $h(t) = \infty$ which is a contradiction of Part 1. 

            
        \color{black}

\end{enumerate}

\pagebreak

4. Assume $f_n \overset{\mu}{\longrightarrow} f$.  
\begin{itemize}
    \item Prove that $\liminf_n \int \abs{f_n} \; d\mu \geq \int \abs{f} \; d\mu$ 
    
    \color{blue}
        Since $f_n \overset{\mu}{\longrightarrow} f$, we have (by a Theorem from class), that there exists a subsequence $f_{n_k} \to f$ a.e.

        Then, $\abs{f_{n_k}} \to \abs{f}$ a.e. so by Fatou's lemma,
        \[\liminf_n \int \abs{f_{n_k}} \; d\mu \geq \int \liminf_n \abs{f_{n_k}} \; d\mu = \int \abs{f} \; d\mu\]  

        By monotonicity of the integral, we also have 
        \[\liminf_n \int \abs{f_n} \; d\mu \geq \liminf_n \int \abs{f_{n_k}} \; d\mu\] 
        
        Therefore, 
        \[\liminf_n \int \abs{f_n} \; d\mu \geq \int \abs{f} \; d\mu\]
        
    \color{black}

    \item Further assume $\abs{f_n} \leq g \in \L^1$. Prove that $f_n \to f$ in $\L^1$.
    
    \color{blue}
        As before, we have that $f_{n_k} \to f$ a.e. and in particular, $\abs{f_{n_k} - f} \to 0$ a.e.
        
        By assumption, $\abs{f_{n_k}} \leq g \in \L^1$ so by the Dominated Convergence Theorem,
        \[\lim \int \abs{f_{n_k} - f} \; d\mu = \int \lim \abs{f_{n_k} - f} \; d\mu = \int 0 \; d\mu= 0\]

        Therefore, $f_{n_k} \to f$ in $\L^1$. 

        Suppose now that $f_{n} \not\to f$ in $\L^1$. Then, $\forall \ep > 0$, $\exists f_{n_i}$ for infinitely many $n_i$ such that 
        \[\int \abs{f_{n_i}- f} \; d\mu \geq \ep\]

        But $f_{n_i} \overset{\mu}{\longrightarrow} f$ so it also has a subsequence $f_{n_{i_j}} \to f$ in $\L^1$. But then $\abs{f_{n_{i_j}} - f} \to 0$ a.e. and $\abs{f_{n_{i_j}}} \leq g \in \L^1$ so $\abs{f_{n_{i_j}} - f} \leq 2g$, hence 
        \[\lim \inf \abs{f_{n_{i_j}}- f} \; d\mu =0\]
        which is a contradiction. 
    \color{black}

\end{itemize}

\end{document}