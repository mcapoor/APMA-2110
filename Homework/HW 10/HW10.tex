\documentclass[12pt]{article}
\usepackage[utf8]{inputenc}
\usepackage{geometry}
\geometry{letterpaper, margin=0.25in}
\usepackage{graphicx} 
\usepackage{parskip}
\usepackage{booktabs}
\usepackage{array} 
\usepackage{paralist} 
\usepackage{verbatim}
\usepackage{subfig}
\usepackage{fancyhdr}
\usepackage{sectsty}
\usepackage[shortlabels]{enumitem}

\pagestyle{fancy}
\renewcommand{\headrulewidth}{0pt} 
\lhead{}\chead{}\rhead{}
\lfoot{}\cfoot{\thepage}\rfoot{}

%%% ToC (table of contents) APPEARANCE
\usepackage[nottoc,notlof,notlot]{tocbibind} 
\usepackage[titles,subfigure]{tocloft}
\renewcommand{\cftsecfont}{\rmfamily\mdseries\upshape}
\renewcommand{\cftsecpagefont}{\rmfamily\mdseries\upshape} %

\usepackage{amsmath}
\usepackage{amssymb}
\usepackage{mathtools}
\usepackage{empheq}
\usepackage{xcolor}
\usepackage{bbm}
\usepackage{tikz}
\usepackage{pgfplots}
\usepackage{tikz-cd}
\pgfplotsset{compat=1.18}

\newcommand{\ans}[1]{\boxed{\text{#1}}}
\newcommand{\vecs}[1]{\langle #1\rangle}
\renewcommand{\hat}[1]{\widehat{#1}}

\renewcommand{\P}{\mathbb{P}}
\newcommand{\R}{\mathbb{R}}
\newcommand{\E}{\mathbb{E}}
\newcommand{\Z}{\mathbb{Z}}
\newcommand{\N}{\mathbb{N}}
\newcommand{\Q}{\mathbb{Q}}
\newcommand{\C}{\mathbb{C}}

\newcommand{\ind}{\mathbbm{1}}
\newcommand{\qed}{\quad \blacksquare}

\newcommand{\brak}[1]{\left\langle #1 \right\rangle}
\newcommand{\bra}[1]{\left\langle #1 \right\vert}
\newcommand{\ket}[1]{\left\vert #1 \right\rangle}

\newcommand{\abs}[1]{\left\vert #1 \right\vert}
\newcommand{\mfX}{\mathfrak{X}}
\newcommand{\ep}{\varepsilon}

\newcommand{\Ec}{\mathcal{E}}
\newcommand{\A}{\mathcal{A}}
\newcommand{\F}{\mathcal{F}}
\newcommand{\Cc}{\mathcal{C}}
\newcommand{\B}{\mathcal{B}}
\newcommand{\M}{\mathcal{M}}
\newcommand{\X}{\chi}
\renewcommand{\L}{\mathcal{L}}

\newcommand{\sub}{\subseteq}
\newcommand{\st}{\text{ s.t. }}
\newcommand{\card}{\text{card }}
\renewcommand{\div}{\vspace*{10pt}\hrule\vspace*{10pt}}
\newcommand{\surj}{\twoheadrightarrow}
\newcommand{\inj}{\hookrightarrow}
\newcommand{\biject}{\hookrightarrow \hspace{-8pt} \rightarrow}
\renewcommand{\bar}[1]{\overline{#1}}
\newcommand{\overcirc}[1]{\overset{\circ}{#1}}
\newcommand{\diam}{\text{diam }}

\renewcommand{\Re}{\text{Re}\,}
\renewcommand{\Im}{\text{Im}\,}
\newcommand{\sign}{\text{sign}\,}

\newcommand*{\tbf}[1]{\ifmmode\mathbf{#1}\else\textbf{#1}\fi}

\usepackage{tcolorbox}
\tcbuselibrary{breakable, skins}
\tcbset{enhanced}
\newenvironment*{tbox}[2][gray]{
    \begin{tcolorbox}[
        parbox=false,
        colback=#1!5!white,
        colframe=#1!75!black,
        breakable,
        title={#2}
    ]}
    {\end{tcolorbox}}

\newenvironment*{exercise}[1][red]{
    \begin{tcolorbox}[
        parbox=false,
        colback=#1!5!white,
        colframe=#1!75!black,
        breakable
    ]}
    {\end{tcolorbox}}

\newenvironment*{proof}[1][blue]{
\begin{tcolorbox}[
    parbox=false,
    colback=#1!5!white,
    colframe=#1!75!black,
    breakable
]}
{\end{tcolorbox}}

\title{APMA 2110 - Homework 10}
\author{Milan Capoor}
\date{27 Nov 2024}

\begin{document}
\maketitle

1. Assume the validity of the change of variables formula for the Riemann integral. Let $f$ be Riemann integrable in $\R^n$ and show that $f$ is Lebesgue integrable. 

Use this fact to give an alternative proof of the change of variables formula for $f \in \L^1(\R^n)$. In particular, prove the integration in polar coordinates for $f\in \L^1(\R^n)$:
\[\int_{\R^n} f = \int_0^{\infty} \int_{S^{n-1}} f(rx') r^{n-1} \;d\sigma(x') \; dr\] 
where $\sigma = \sigma_{n-1}$ is the surface measure on the unit sphere $S^{n-1}$.

    \color{blue}
        We present an alternative proof of the change of variables formula for $f \in \L^1(\R^n)$.

        STEP 1. Assume that for $\phi: E \to \R^n$ a $C^1$ diffeomorphism and $f: \phi(E) \to \R^n$ Riemann integrable,  
        \[\int_{\phi(E)} f(y)\; dy = \int_{E} (f\circ \phi)(x) \abs{\det D_x\phi} \; dx\]

        STEP 2. For smooth functions in $\R^n$, the Riemann integral is equivalent to the Lebesgue integral.

        \begin{proof}

            \emph{Proof:} Suppose $f$ is a smooth function in $\R^n$.

            By definition, $f$ is continuous. Trivially, $\{x: f(x) \text{ not continuous}\}$ has measure zero, so by a Theorem from class, $f$ is Riemann integrable. 

            Note that we showed in class that the Lebesgue and Riemann integrals are equivalent in $\R$. We can extend this to $\R^n$ by considering the product measure $m^n = m \times \dots \times m$ on $\R^n$ and applying Fubini: 
            \begin{align*}
                \int_{\R^n} f\; dm^n &= \int_{\R} \int_{\R} \dots\int_{\R} f \;dm(x_1), dm(x_2)\dots dm(x_n)\\ 
                &= \int_{-\infty}^{\infty} \int_{-\infty}^{\infty} \dots \int_{-\infty}^{\infty} f \;dx_1 dx_2 \dots dx_n
            \end{align*}
            
        \end{proof}

        STEP 3. By the approximation of $\L^1(m)$ functions (shown in class), there exists a smooth function $g$ such that $\abs{\abs{f - g}}_{\L^1} < \ep$.

        Hence, up to order $\ep$, the change of variables formula holds for $f \in \L^1(\R^n)$.

        \div 

        Denote the unit sphere $S^{n-1} = \{x \in \R^n: \abs{x} = 1\}$. Then for $x \in \R^n \setminus \{0\}$, the polar coordinates of $x$ are given by:
        \[\begin{cases}
            r = \abs{x} \in (0, \infty)\\ 
            x' = \frac{x}{\abs{x}} \in S^{n-1}
        \end{cases}\]

        Consider the map $\Phi: (0, \infty) \times S^{n-1} \to \R^n\setminus \{0\}$ given by $\Phi(r, x') = rx'$.

        By the change of variables formula above,
        \begin{align*}
            \int_{\R^n} f &= \int_{(0, \infty) \times S^{n-1}} f(\Phi(r, x')) \abs{\det D\Phi(r, x')} \; d(\sigma \times r)\\ 
            &= \int_0^{\infty} \int_{S^{n-1}} f(rx') r^{n-1} \; d\sigma(x') \; dr \qquad (\text{Tonelli})
        \end{align*}

    \color{black}
\pagebreak 

2. Let $\nu$ be a signed measure on $(X, \M)$. Prove:
\begin{itemize}
    \item If $E_j$ is an increasing sequence of sets in $\M$, $\nu\left(\bigcup_{j=1}^\infty E_j\right) = \lim_{j\to\infty} \nu(E_j)$.
    \item If $E_j$ is a decreasing sequence of sets in $\M$ and $\nu(E_1) < \infty$, then $\nu\left(\bigcap_{j=1}^\infty E_j\right) = \lim_{j\to\infty} \nu(E_j)$.
\end{itemize}

    \color{blue}
        We would like to rewrite $\bigcup_j E_j$ as a union of disjoint sets to take advantage of countable additivity. 
        
        WLOG define $E_0 = \emptyset$. Then, because $E_n \sub E_{n+1}$,
        \begin{align*}
            E_1 &= E_1 \setminus \emptyset\\ 
            E_2 &= E_1 \cup (E_2 \setminus E_1)\\ 
            E_3 &= E_2 \cup (E_3 \setminus E_2)\\ 
                &= (E_1 \setminus \emptyset) \cup (E_2 \setminus E_1) \cup (E_3 \setminus E_2) 
        \end{align*}
    
        Hence, inductively define 
        \[E_n = \bigcup_{k=0}^{n-1} E_{k+1} \setminus E_k\]
        so 
        \[\bigcup_{n=1}^\infty E_n = \bigcup_{n=0}^\infty (E_n \setminus E_{n-1})\]

        Then, by countable additivity,
        \begin{align*}
            \nu\left(\bigcup_{n=1}^\infty E_n\right) &= \nu\left(\bigcup_{n=0}^\infty (E_n \setminus E_{n-1})\right)\\ 
            &= \sum_{n=0}^\infty \nu(E_n \setminus E_{n-1})\\ 
            &= \lim_{N\to\infty} \sum_{n=0}^N \nu(E_n \setminus E_{n-1})\\ 
            &= \lim_{n \to \infty} \nu(E_n)
        \end{align*}

        \div 

        Let $E_1 \supset E_2 \supset \dots$ and $\nu(E_1) < \infty$. Define $F_j = E_1 \setminus E_j$. Clearly, $F_n \sub F_{n+1}$. 

        By the previous part,
        \begin{align*}
            \nu\left(\bigcup_{n=1}^\infty F_n\right) &= \lim_{n \to \infty} \nu(F_n)\\ 
            F_j = E_1 \setminus E_j &\implies \bigcup_{j=1}^n F_j = E_1 \setminus \bigcup_{j=1}^n E_j\\ 
            \nu\left(E_1 \setminus \bigcap_{n=1}^\infty E_n\right) &= \lim_{n \to \infty} \nu(E_1 \setminus \bigcap_{j=1}^n E_j )\\
        \end{align*}

        In a signed measure, we do not have monotonicity but we do have that for $E \sub F$,
        \[\nu(F) = \nu(E \cup F \setminus E) = \nu(E) + \nu(F \setminus E)\]

        \begin{proof}
            \emph{Proof:} Consider $F \setminus E = F \cap E^c$. But $E \cap (F \cap E^c) = \emptyset$ so 
            \[\mu(F) = \mu(E \cup F \setminus E) = \mu(E) + \mu(F \setminus E)\]
        \end{proof}

        Because $\nu(E_1) < \infty$,
        \begin{align*}
            \nu(E_1) - \nu\left(\bigcap_{n=1}^\infty E_n\right) &= \lim_{n \to \infty} \left[\nu(E_1) - \nu\left(\bigcap_{j=1}^n E_j \right)\right]\\ 
                &= \lim_{n \to \infty} \left[\nu(E_1) - \nu(E_n)\right]\\ 
                &= \nu(E_1) - \lim_{n \to \infty} \nu(E_n)\\
            \nu\left(\bigcap_{n=1}^\infty E_n\right) &= \lim_{n \to \infty} \nu(E_n) \qed
        \end{align*}

    \color{black}

\pagebreak 

3. If $\nu$ is a signed measure and $\lambda, \mu$ are positive measures such that $\nu = \lambda - \mu$, show $\lambda \geq \nu^+$ and $\mu \geq \nu^-$.

    \color{blue}
        $\nu$ is a signed measure, so by Hahn decomposition, $\exists P, N$ unique with $P$ positive and $N$ negative such that
        \[P \cup N = X, \; P \cap N = \emptyset\]

        Define 
        \[\nu^+(A) = \nu(A \cap P), \; \nu^-(A) = -\nu(A \cap N)\]

        Then, $\nu = \nu^+ - \nu^-$.

        First consider $\lambda$. Because $\lambda, \mu$ are positive measures, we have $\lambda \geq \lambda - \mu = \nu$. 

        In particular, $\forall A \in \M$, 
        \[\lambda(A) \overset{*}{\geq} \lambda(A \cap P) \geq \nu(A \cap P) = \nu^+(A)\]
        where $\overset{*}{\geq}$ follows from the fact that $\lambda$ is a positive measure (and hence monotonic).

        Now consider $\mu$. We have $-\nu = \mu - \lambda$ so in particular, $\mu \geq -\nu$. Hence, for any $A \in \M$,
        \[\mu(A) \geq \mu(A \cap N) \geq -\nu(A \cap N) = \nu^-(A)\]
        again by monotonicity of $\mu$. $\qed$
       
    \color{black}


\pagebreak

4. Suppose that $\nu$ is a signed measure on $(X, \M)$ and $E \in \M$. Prove 
\begin{enumerate}
    \item \begin{align*}
        \nu^+(E) &= \sup\{\nu(F): F \sub E, F \in \M\} \\
        \nu^-(E) &= -\inf\{\nu(F): F \sub E, F \in \M\} 
    \end{align*}

    \item 
    \[\abs{\nu}(E) = \sup \left\{\sum_{j=1}^n \abs{\nu(E_j)} \; \bigg\vert \; n \in \N,\; E_1\, \dots\, E_n  \text{ disjoint}, \; \bigcup_1^n E_j = E\right\}\]
\end{enumerate}

    \color{blue}
        Let $X = P \cup N$ be the Hahn decomposition of $X$ and 
        \begin{align*}
            \nu^+(E) &= \nu(E \cap P)\\ 
            \nu^-(E) &= -\nu(E \cap N)
        \end{align*}
        
        Define $\bar F = \arg\sup \{\nu(F) : F \sub E, \; F \in \M\}$.

        STEP 1. $\bar F$ is a positive set and $E \setminus \bar F$ is a negative set. 

        \begin{proof}
            \emph{Proof:} Let $\bar F = A \cup B$ where $A$ is a positive set and $B$ is a negative set such that $A \cap B = \emptyset$ (guaranteed by Hahn). 
            
            Suppose, for contradiction, that $\bar F$ is not positive, i.e. $B \neq \emptyset$ 

            We know $\nu(\bar F) = \nu(\bar F \setminus B) + \nu(B)$ (countable additivity), so in particular, $\nu(\bar F \setminus B) \geq \nu(\bar F) $ (as $\nu(B)< 0$). But this contradicts the maximality of $\bar F$. 

            Hence, $\bar F$ is a positive set.

            \div 

            Similarly, suppose $E \setminus \bar F$ is not negative, i.e. it contains some positive set $C$. But then, $C$ and $\bar F$ are disjoint so by countable additivity, 
            \[\nu(\bar F \cup C) = \nu(\bar F) + \nu(C) \geq \nu(\bar F)\]
            which again contradicts the maximality of $\bar F$.

        \end{proof}

        STEP 2. $\nu(\bar F) = \nu(E \cap P) = \nu^+(E)$. 

        \begin{proof}
            \emph{Proof:} In fact, we have the strictly stronger claim that $\bar F = E \cap P$: clearly, $\bar F \sub E$ and further, $\bar F$ is positive (by Step 1), so $\bar F \sub E \cap P$. 

            Then, suppose $\exists D \in (E \cap P) \setminus \bar F$. But then $D$ is a positive set in $E$ which is disjoint from $\bar F$ so $\nu(D \cup \bar F) = \nu(D) + \nu(\bar F) > \nu(\bar F)$, contradicting the maximality of $\bar F$. Hence, $E \cap P \sub \bar F$

            Certainly, $\bar F = E \cap P \implies \nu(\bar F) = \nu(E \cap P) = \nu^+(E)$, by definition.  
        \end{proof}        

        $\nu^-$ follows by similar argument on $\underbar F = \arg\inf \{\nu(F) : F \sub E, \; F \in \M\}$. $\qed$

        \div

        Once again, let $X = P \cup N$ be the Hahn decomposition of $X$ and let $\nu^+(E) = \nu(E \cap P)$ and $\nu^-(E) = -\nu(E \cap N)$.

        Notice that $E = (E \cap P) \cup (E \cap N)$ and $E \cap P, E \cap N$ are disjoint, so, by definition of the supremum,
        \begin{align*}
            \sup \left\{\sum_{j=1}^n \abs{\nu(E_j)} \; \bigg\vert \; n \in \N,\; E_1\, \dots\, E_n  \text{ disjoint}, \; \bigcup_1^n E_j = E\right\} &\geq \abs{\nu(E \cap P)} + \abs{\nu(E \cap N)}\\ 
            &= \nu^+(E) + \abs{-\nu^-(E)}\\ 
            &= \nu^+(E) + \nu^-(E)\\ 
            &= \abs{\nu}(E)
        \end{align*}

        Conversely, let $E = \bigcup_{j=1}^n E_j$ for $E_j$ disjoint. Then 
        \begin{align*}
            \abs{\nu}(E) &= \abs{\nu}\left(\bigcup_{j=1}^n E_j\right)\\
                &= \sum_{j=1}^n \abs{\nu}(E_j) \qquad (\text{by countable additivity})\\
                &= \sum_{j=1}^n \nu^+(E_j) + \nu^-(E_j)\\
                &\geq \sum_{j=1}^n \abs{\nu^+(E_j) - \nu^-(E_j)}\\ 
                &= \sum_{j=1}^n \abs{\nu(E_j)}
        \end{align*}
        Taking the supremum over all such decompositions, we have the desired result. $\qed$
        


    \color{black}

\end{document}