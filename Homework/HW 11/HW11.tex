\documentclass[12pt]{article}
\usepackage[utf8]{inputenc}
\usepackage{geometry}
\geometry{letterpaper, margin=0.25in}
\usepackage{graphicx} 
\usepackage{parskip}
\usepackage{booktabs}
\usepackage{array} 
\usepackage{paralist} 
\usepackage{verbatim}
\usepackage{subfig}
\usepackage{fancyhdr}
\usepackage{sectsty}
\usepackage[shortlabels]{enumitem}

\pagestyle{fancy}
\renewcommand{\headrulewidth}{0pt} 
\lhead{}\chead{}\rhead{}
\lfoot{}\cfoot{\thepage}\rfoot{}

%%% ToC (table of contents) APPEARANCE
\usepackage[nottoc,notlof,notlot]{tocbibind} 
\usepackage[titles,subfigure]{tocloft}
\renewcommand{\cftsecfont}{\rmfamily\mdseries\upshape}
\renewcommand{\cftsecpagefont}{\rmfamily\mdseries\upshape} %

\usepackage{amsmath}
\usepackage{amssymb}
\usepackage{mathtools}
\usepackage{empheq}
\usepackage{xcolor}
\usepackage{bbm}
\usepackage{tikz}
\usepackage{pgfplots}
\usepackage{tikz-cd}
\pgfplotsset{compat=1.18}

\newcommand{\ans}[1]{\boxed{\text{#1}}}
\newcommand{\vecs}[1]{\langle #1\rangle}
\renewcommand{\hat}[1]{\widehat{#1}}

\renewcommand{\P}{\mathbb{P}}
\newcommand{\R}{\mathbb{R}}
\newcommand{\E}{\mathbb{E}}
\newcommand{\Z}{\mathbb{Z}}
\newcommand{\N}{\mathbb{N}}
\newcommand{\Q}{\mathbb{Q}}
\newcommand{\C}{\mathbb{C}}

\newcommand{\ind}{\mathbbm{1}}
\newcommand{\qed}{\quad \blacksquare}

\newcommand{\brak}[1]{\left\langle #1 \right\rangle}
\newcommand{\bra}[1]{\left\langle #1 \right\vert}
\newcommand{\ket}[1]{\left\vert #1 \right\rangle}

\newcommand{\abs}[1]{\left\vert #1 \right\vert}
\newcommand{\norm}[1]{\left\vert\left\vert #1 \right\vert\right\vert}

\newcommand{\mfX}{\mathfrak{X}}
\newcommand{\ep}{\varepsilon}

\newcommand{\Ec}{\mathcal{E}}
\newcommand{\A}{\mathcal{A}}
\newcommand{\F}{\mathcal{F}}
\newcommand{\Cc}{\mathcal{C}}
\newcommand{\B}{\mathcal{B}}
\newcommand{\M}{\mathcal{M}}
\newcommand{\X}{\chi}
\renewcommand{\L}{\text{L}}

\newcommand{\sub}{\subseteq}
\newcommand{\st}{\text{ s.t. }}
\newcommand{\card}{\text{card }}
\renewcommand{\div}{\vspace*{10pt}\hrule\vspace*{10pt}}
\newcommand{\surj}{\twoheadrightarrow}
\newcommand{\inj}{\hookrightarrow}
\newcommand{\biject}{\hookrightarrow \hspace{-8pt} \rightarrow}
\renewcommand{\bar}[1]{\overline{#1}}
\newcommand{\overcirc}[1]{\overset{\circ}{#1}}
\newcommand{\diam}{\text{diam }}

\renewcommand{\Re}{\text{Re}\,}
\renewcommand{\Im}{\text{Im}\,}
\newcommand{\sign}{\text{sign}\,}

\newcommand*{\tbf}[1]{\ifmmode\mathbf{#1}\else\textbf{#1}\fi}

\usepackage{tcolorbox}
\tcbuselibrary{breakable, skins}
\tcbset{enhanced}
\newenvironment*{tbox}[2][gray]{
    \begin{tcolorbox}[
        parbox=false,
        colback=#1!5!white,
        colframe=#1!75!black,
        breakable,
        title={#2}
    ]}
    {\end{tcolorbox}}

\newenvironment*{exercise}[1][red]{
    \begin{tcolorbox}[
        parbox=false,
        colback=#1!5!white,
        colframe=#1!75!black,
        breakable
    ]}
    {\end{tcolorbox}}

\newenvironment*{proof}[1][blue]{
\begin{tcolorbox}[
    parbox=false,
    colback=#1!5!white,
    colframe=#1!75!black,
    breakable
]}
{\end{tcolorbox}}

\title{APMA 2110: Homework 11}
\author{Milan Capoor}
\date{10 Dexember 2024}

\begin{document}
\maketitle
1. Suppose $\mu$ and $\nu$ are $\sigma$-finite measures on $(X, \M)$ with $\nu \ll \mu$, and let $\lambda = \mu + \nu$. If $f = \frac{d\nu}{d\lambda}$, show $0 \leq f < 1$ a.e. and $\frac{d\nu}{d\mu} = \frac{f}{1 - f}$.  

    \color{blue}
        STEP 1. $0 \leq f$ $\mu$-a.e.

        It suffices to show that $E = \{x: f(x) < 0\}$ has measure zero.

        Notice though, for $E_n = \{x: f(x) < -\frac{1}{n}\}$,
        \[E = \{x: f(x) < 0\} = \bigcup_{n=1}^\infty E_n\]
        and since $f = \frac{d\nu}{d\lambda}$,
        \[\nu(E_n) = \int_{E_n} f \; d\lambda < \int_{E_n} -\frac{1}{n} \; d\lambda = -\frac{1}{n} \lambda(E_n)\]

        But $\nu \geq 0$, so
        \[-\frac{1}{n}\lambda(E_n) \geq 0 \implies \lambda(E_n) = \mu(E_n) + \nu(E_n) \leq 0 \implies \mu(E_n) \leq 0\]

        But again $\mu(E_n) \geq 0$, so $\mu(E_n) = 0$ and hence $\mu(E) = 0$.
        
        STEP 2. $f < 1$ $\mu$-a.e.

        Now consider $F = \{x: f(x) \geq 1\}$. By $\sigma$-finiteness of $\nu$, $\exists \{F_n\}$ such that $F = \bigcup_{n=1}^\infty F_n$ and $\nu(F_n) < \infty$ for all $n$.

        Since $f = \frac{d\nu}{d\lambda}$,
        \[\nu(F_n) = \int_{F_n} f \; d\lambda \geq \int_{F_n} 1 \; d\lambda = \lambda(F_n) = \mu(F_n) + \nu(F_n)\]

        Since $\nu(F_n) < \infty$, $0 \geq \mu(F_n) \implies \mu(F_n) = 0 \implies \mu(F) = 0$ and $f < 1$ $\mu$-a.e.

        \div 

        $\mu \leq \lambda \implies \mu \ll \lambda$ and $\nu \ll \mu$ by assumption so by the chain rule, 
        \[\frac{d\nu}{d\mu} = \frac{d\nu}{d\lambda} \cdot \frac{d\lambda}{d\mu} = f \cdot \frac{d\lambda}{d\mu}\]

        Hence, it suffices to show that $\frac{d\lambda}{d\mu} = \frac{1}{1 - f}$. 

        But in fact, since $\nu \ll \mu$ by assumption, 
        \[\mu(E) = 0 \implies \nu(E) = 0 \implies \nu(E) + \mu(E) = \lambda(E) = 0\] 
        so we also have that $\lambda \ll \mu$.

        In particular, this means that 
        \[\frac{d\lambda}{d\mu} \cdot \frac{d\mu}{d\lambda} = 1 \; \text{a..e}\]
        and it in fact suffices to show that $\frac{d\mu}{d\lambda} = 1 - f$. 

        Consider
        \begin{align*}
            \mu(E) + \nu(E) &= \lambda(E)\\ 
                 &= \int_E 1 \; d\lambda\\ 
                 &= \int_E (1 - f) \; d\lambda + \int_E f \; d\lambda\\
                    &= \int_E (1 - f) \; d\lambda + \nu(E)
        \end{align*} 
        and since $\nu$ is $\sigma$-finite, 
        \[\mu(E) = \int_E (1 - f)\; d\lambda \implies \frac{d\mu}{d\lambda} = 1 - f\]
        exactly as desired. $\qed$
    \color{black}


\pagebreak

2. Let $f \in \L^1(\R^n)$ and recall the average function 
\[A_r f(x) = \frac{1}{m(B(r, x))} \int_{B(r, x)} f(y)\; dy\]

Prove that 
\[\lim_{r \to 0} \norm{A_r f - f}_{\L^1(\R^n)} \to 0\]

Deduce there is a subsequence $r_n \to 0$ such that $A_{r_n} f \to f$ a.e. 

    \color{blue}
        Let $\ep > 0$. 
    
        Since $f \in \L^1$, by a theorem from class, we may approximate by a continuous integrable function with compact support $g$ such that 
        \[\int_{\R^n} \abs{f - g} \; dy < \frac{\ep}{3}\]

        By the triangle inequality, 
        \begin{align*}
            \lim_{r \to 0} \norm{A_r f - f}_{\L^1} &= \lim_{r \to 0} \int_{\R^n} \abs{A_r f - f}\; dm\\ 
                &= \lim_{r \to 0} \int_{\R^n} \abs{A_r f - A_r g + A_r g - g + g - f}\; dm\\
                &\leq \lim_{r \to 0} \int_{\R^n} \abs{A_r f - A_r g}\; dm + \lim_{r \to 0} \int_{\R^n} \abs{A_r g - g}\; dm + \lim_{r \to 0} \int_{\R^n} \abs{g - f}\; dm
        \end{align*}

        For fixed $r > 0$, by continuity of $g$, $\abs{y - x} < r$ implies $\abs{g(y) - g(x)} < \delta$, so 
        \[\abs{A_r g - g} = \abs{\frac{1}{m(B(r, x))}\int_{B(r, x)} [g(y) - g(x)] \; dy} < \delta\]
        
        Now since $g$ has compact support (say on some set $K \sub \R^n$), 
        \[\int_{\R^n} \delta \; dm = \int_{K} \delta \; dm = \delta m(K) < \infty\]
        since the Lebesgue measure is $\sigma$-finite so letting $\delta = \frac{\ep}{3m(K)}$, $\lim_{r \to 0} \int_{\R^n} \abs{A_r g - g} < \frac{\ep}{3}$.

        Further, by definition,
        \begin{align*}
            \int_{\R^n} \abs{A_r f - A_r g} &= \int_{\R^n}\abs{\frac{1}{m(B(r, x))} \int_{B(r, x)} f(y) - g(y)\; dy}\; dx\\
                &\leq \int_{\R^n}\frac{1}{m(B(r, x))} \int_{B(r, x)} \abs{f(y) - g(y)}\; dy\;dx\\ 
                &\leq \frac{1}{m(B(r, x))} \int_{\R^n} \int_{B(r, x)} \abs{f(y) - g(y)}\; dy\; dx \qquad (\text{since }m(B(r, x)) \text{ is constant WRT } x)\\
                &= \frac{1}{m(B(r, x))} \int_{\R^n} \int_{\R^n} \abs{f(y) - g(y)} \ind_{B(r, x)}\; dy\; dx\\
                &= \frac{1}{m(B(r, x))} \int_{\R^n} \int_{\R^n} \abs{f(y) - g(y)} \ind_{B(r, x)}\; dx\; dy \qquad (\text{Fubini})\\
                &= \frac{1}{m(B(r, x))} \int_{\R^n} \abs{f(y) - g(y)} \int_{\R^n} \ind_{B(r, x)}\; dx\; dy\\
                &= \frac{1}{m(B(r, x))} \int_{\R^n} \abs{f(y) - g(y)} m(B(r, x))\; dy\\
                &= \int_{\R^n} \abs{f(y) - g(y)}\; dy < \frac{\ep}{3}
        \end{align*}

        Hence, 
        \[\lim_{r \to 0} \norm{A_r f - f}_{\L^1} < \frac{\ep}{3} + \frac{\ep}{3} + \frac{\ep}{3} = \ep\]

        Therefore, $A_r f \to f$ in $\L^1$ (take $r = \frac{1}{n}$ for the limit $n \to \infty$), and by a theorem from class, there is a subsequence $r_n \to 0$ such that $A_{r_n} f \to f$ a.e. $\qed$
        
    
    \color{black}


\pagebreak 

3. Define a variant of the maximal function $H(f)$ in $\R^n$ as 
\[H^* f(x) = \sup \left\{\frac{1}{m(B)} \int_B \abs{f(y)}\; dy \; \bigg\vert \; B \text{ is a ball and } x \in B\right\}\]

Show $Hf \leq H^* f \leq 2^n Hf$. 

    \color{blue}
        Recall that 
        \[H f(x) = \sup_{r > 0} \frac{1}{m(B(r, x))} \int_{B(r, x)} \abs{f(y)}\; dy\]
        where $B(r, x)$ is the n-ball of radius $r$ centered at $x$.

        Pick any $r > 0$. Let $B(r, x)$ be a ball of radius $r$ centered at $x$. Certainly, $x \in B(r, x)$ so by definition of $H^* f(x)$,
        \[\frac{1}{m(B(r, x))} \int_{B(r, x)} \abs{f(y)}\; dy \leq H^* f(x)\]

        Taking the $\sup$ over $r > 0$, we have that $H f(x) \leq H^* f(x)$.

        Conversely, for any ball $B_r$ of radius $r$ with $x \in B_r$, $B_r \sub B(2r, x)$ (the ball of radius $2r$ centered at $x$). Hence,
        \[\frac{1}{m(B_r)} \int_{B_r} \abs{f(y)}\; dy \leq \frac{1}{m(B_r)} \int_{B(2r, x)} \abs{f(y)}\; dy = \frac{2^n}{m(B(2r, x))} \int_{B(2r, x)} \abs{f(y)}\; dy \leq 2^n H f\]

        Again taking the $\sup$ on both sides, $H^* f(x) \leq 2^n H f(x)$. $\qed$
    \color{black}


\pagebreak

4. Assume $\mu$ is a positive measure and $f_n \to f$ in $\L^1(\mu)$. Prove that $\forall \ep > 0$, $\exists \delta  >0$ such that $\abs{\int_E f_n \; d\mu} < \ep$ for all $n$ and any $\mu(E) < \delta$. 

    \color{blue}
        Let $\ep > 0$. 
        
        By a lemma from class, 
        \[\abs{\int_E f_n \; d\mu} \leq \int_E \abs{f_n} \; d\mu\]
        for any set $E \sub X$. 

        By the triangle inequality,
        \[\int_E \abs{f_n} \; d\mu = \int_E \abs{f_n - f + f} \; d\mu \leq \int_E \abs{f_n - f} \; d\mu + \int_E \abs{f} \; d\mu\]

        Since $f \in \L^1$, $\int \abs{f} \; d\mu = M < \infty$. Hence, for $\delta_1 = \frac{\ep}{2M}$ and $\mu(E) < \delta_1$, 
        \[\int_E \abs{f} \; d\mu < \frac{\ep}{2} \]
    
        Further, since $f_n \to f$ in $\L^1$, $\exists N$ such that $\forall n \geq N$ and any measurable set $E$,
        \[\int_E \abs{f_n - f} \; d\mu < \frac{\ep}{2}\] 

        It remains to show that $\int_E \abs{f_n - f} \; d\mu < \frac{\ep}{2}$ for $n < N$.     

        Consider the finite set $\{f_n\}_{n=1}^{N-1} \sub \L^1$. By a corollary from class, $\forall \ep > 0$, $\exists \delta_{f_n} > 0$ such that $\mu(E) < \delta$ implies $\int_E \abs{f_n} \; d\mu < \frac{\ep}{2}$. 
        
        Let $\delta_2 = \max\{\delta_{f_1}, \ldots, \delta_{f_{N-1}}\}$. Then, $\mu(E) < \delta_2$ implies
        \[\int_E \abs{f_n} \; d\mu < \frac{\ep}{2}\]
        for all $n < N$.

        Taking $\delta = \max\{\delta_1, \delta_2\}$, we have that $\mu(E) < \delta$ implies
        \[\int_E \abs{f_n} \; d\mu < \frac{\ep}{2} + \frac{\ep}{2} = \ep \qed\]
        
    \color{black}


\end{document}