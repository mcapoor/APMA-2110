\documentclass[12pt]{article} 
\usepackage[utf8]{inputenc}
\usepackage{geometry}
\geometry{letterpaper,
    margin=0.5in
}
\usepackage{graphicx} 
\usepackage{parskip}
\usepackage{booktabs}
\usepackage{array} 
\usepackage{paralist} 
\usepackage{verbatim}
\usepackage{subfig}
\usepackage{fancyhdr}
\usepackage{sectsty}
\usepackage[shortlabels]{enumitem}

\pagestyle{fancy}
\renewcommand{\headrulewidth}{0pt} 
\lhead{}\chead{}\rhead{}
\lfoot{}\cfoot{\thepage}\rfoot{}


%%% ToC (table of contents) APPEARANCE
\usepackage[nottoc,notlof,notlot]{tocbibind} 
\usepackage[titles,subfigure]{tocloft}
\renewcommand{\cftsecfont}{\rmfamily\mdseries\upshape}
\renewcommand{\cftsecpagefont}{\rmfamily\mdseries\upshape} %

\usepackage{amsmath}
\usepackage{amssymb}
\usepackage{mathtools}
\usepackage{empheq}
\usepackage{xcolor}

\usepackage{tikz}
\usepackage{pgfplots}
\usepackage{tikz-cd}
\pgfplotsset{compat=1.18}

\newcommand{\ans}[1]{\boxed{\text{#1}}}
\newcommand{\vecs}[1]{\langle #1\rangle}
\renewcommand{\hat}[1]{\widehat{#1}}
\renewcommand{\P}{\mathbb{P}}
\newcommand{\R}{\mathbb{R}}
\newcommand{\E}{\mathbb{E}}
\newcommand{\Z}{\mathbb{Z}}
\newcommand{\N}{\mathbb{N}}
\newcommand{\Q}{\mathbb{Q}}
\newcommand{\ind}{\mathbbm{1}}
\newcommand{\qed}{\quad \blacksquare}
\newcommand{\brak}[1]{\left\langle #1 \right\rangle}
\newcommand{\bra}[1]{\left\langle #1 \right\vert}
\newcommand{\ket}[1]{\left\vert #1 \right\rangle}
\newcommand{\abs}[1]{\left\vert #1 \right\vert}
\newcommand{\mfX}{\mathfrak{X}}
\newcommand{\ep}{\varepsilon}

\newcommand{\Ec}{\mathcal{E}}
\newcommand{\Fc}{\mathcal{F}}
\newcommand{\sub}{\subseteq}
\newcommand{\st}{\text{ s.t. }}
\renewcommand{\div}{\vspace*{10pt}\hrule\vspace*{10pt}}
\newcommand{\A}{\mathcal{A}}
\newcommand{\B}{\mathcal{B}}
\newcommand{\M}{\mathcal{M}}

\usepackage{tcolorbox}
\tcbuselibrary{breakable, skins}
\tcbset{enhanced}
\newenvironment*{tbox}[2][gray]{
    \begin{tcolorbox}[
        parbox=false,
        colback=#1!5!white,
        colframe=#1!75!black,
        breakable,
        title={#2}
    ]}
    {\end{tcolorbox}}

\newenvironment*{proof}[1][blue]{
        \begin{tcolorbox}[
            parbox=false,
            colback=#1!5!white,
            colframe=#1!75!black,
            coltext=#1,
            breakable
        ]}
        {\end{tcolorbox}}

\title{APMA 2110 - Homework 3}
\author{Milan Capoor}
\date{}

\begin{document}
\maketitle

1. An algebra $\A$ is a $\sigma$-algebra iff $\{E_j\}_{1}^{\infty} \in \A$ and $E_1 \sub E_2 \sub \dots$, then $\bigcup_{j=1}^{\infty} E_j \in \A$.

    \color{blue}
        Suppose $\A$ is a $\sigma$-algebra. Then by definition, $\A$ is closed under countable unions. Trivially, $\bigcup_{j=1}^{\infty} E_j \in \A$ since $E_j \in \A$ for countably many $j$. 

        \div

        Conversely, suppose $\{E_j\}_{1}^{\infty} \in \A$ and $E_1 \sub E_2 \sub \dots$, then $\bigcup_{j=1}^{\infty} E_j \in \A$. We want to show that $\A$ is a $\sigma$-algebra. Clearly, $\A$ is closed under countable unions. So it suffices to show that $\A$ is closed under complements. 
        
        Take $E_1 \in \A$. Then $E_1^c = (E_1^c \cap E_2) \cup E_2^c$. Certainly $E_1^c \cap E_2 \in \A$. Further, $E_2^c \in \A$ since $\A$ is an algebra and closed under complements for finitely many elements. Since $\A$ is closed under finite disjoint unions, $E_1^c \in \A$. 
        
        Suppose $E_1^c, \dots, E_{n}^c \in \S$. Let $E_n \in \A$. We want to show that $E_n^c \in \A$. Notice that 
        \begin{align*}
            E_n^c &= E_{n-1}^c \setminus (E_n \cap E_{n-1}^c)\\ 
                &= E_{n-1}^c \cap (E_{n} \cap E_{n-1}^c)^c\\ 
                &= (E_{n-1}^c \cap E_{n}^c) \cup (E_{n-1}^c \cap E_{n-1}^c)\\ 
                &= (E_{n-1}^c \cap E_{n}^c) \cup E_{n-1}^c\\ 
                &\sub E_{n-1}^c \cup E_{n-1}^c\\ 
                &= E_{n-1}^c
        \end{align*}
        but by assumption, $E_{n-1}^c \in \A$ so $E_n^c \in \A$. 
        
    \color{black}


\pagebreak

2. Prove the Borel set of $\R$, $\B_{\R}$ is generated by each of the following:
\begin{itemize}
    \item the half-open intervals $\{(a,b]: a < b\}$ or $\{[a,b): a < b\}$.
    
        \color{blue}
            \textbf{Lemma:} $\Ec \sub \M(\Fc) \implies \M(\Ec) \sub \M(\Fc)$

            \begin{proof}
                \emph{Proof:} By definition, 
                \[\M(\Ec) = \bigcap_{\Ec \in \A} \A\]
                where $\A$ is a $\sigma$-algebra containing $\Ec$.

                By assumption, $\M(\Fc)$ is a $\sigma$-algebra containing $\Ec$. Hence, $\M(\Fc) = \A$ for some $\A$ and $\M(\Ec)$ is the intersection of all $\A$, so $\M(\Ec) \sub \M(\Fc)$.
            \end{proof}

            Let $\Ec = \{(a, b]: a < b\}$. We want to show that $\B_{\R}$ is generated by $\Ec$, i.e. 
            \[\B_{\R} = \M(\Ec) = \bigcap_{\Ec \sub \A} \A \]

            Certainly $\B_{\R} \sub \M(\Ec)$ by the Lemma above because for any open set $O \sub \B_{\R}$, 
            \[O = \bigcup_{i=1}^\infty (a_i, b_i) \sub \bigcup_{i=1}^\infty (a_i, b_i]\]
            which is a countable union so $\B_{\R} \sub \M(\Ec)$.
            
            It remains to show that $\M(\Ec)\sub \B_{\R}$. 

            We claim
            \[(a, b] = \bigcap_{n=1}^{\infty} (a, b + \frac{1}{n})\]

            \begin{proof}
                \emph{Proof:} Let $(a, b] = (a, b) \cup \{b\}$. Certainly $(a, b) \in \bigcap_{n=1}^{\infty} (a, b + \frac{1}{n})$. Further, 
                \[\lim_{n \to \infty} b + \frac{1}{n} = b \implies b \in (a, b + \frac{1}{n})\] 
                for sufficiently large $n$. Hence $b \in \bigcap_{n=1}^{\infty} (a, b + \frac{1}{n})$ and $(a, b] \sub \bigcap_{n=1}^{\infty} (a, b + \frac{1}{n})$. 

                Conversely, $b \leq b + \frac{1}{n}$ for all $n \in \N$ so $(a, b + \frac{1}{n}) \sub (a, b]$ for all $n \in \N$. Hence $\bigcap_{n=1}^{\infty} (a, b + \frac{1}{n}) \sub (a, b]$.
            \end{proof}

            Then, any $X \in \M(\Ec)$ is a countable intersection of open sets in $\R$, so $X \in \B_{\R} \implies \M(\Ec) \sub \B_{\R}$.     
            
            The argument for $\{[a, b): a < b\}$ is similar with 
            \[[a, b) = \bigcap_{n=1}^{\infty} (a - \frac{1}{n}, b)\]

        \color{black}

    \item the closed rays $\{[a,\infty): a \in \R\}$ or $\{(-\infty, a]: a \in \R\}$.
    
    \color{blue}
        Let $\Ec = \{[a, \infty): a \in \R\}$.
        
        Because $\Ec$ is generated by closed sets in $\R$, $\M(\Ec) \sub \B_{\R}$. 

        For the reverse inclusion, we want to show that $\B_{\R} \sub \M(\Ec)$.

        We claim 
        \[(a, b) = \bigcap_{n=1}^{\infty} [a - \frac{1}{n}, b)\]

        \begin{proof}
            \emph{Proof:} 
            If $a < x < b$, certainly $a - \frac{1}{n} \leq x < b$ for all $n \in \N$. Hence $(a, b) \sub \bigcap_{n=1}^{\infty} [a - \frac{1}{n}, b)$.

            Conversely, if $x \in \bigcap_{n=1}^{\infty} [a - \frac{1}{n}, b)$, then $a - \frac{1}{n} \leq x < b$ for all $n \in \N$. But $a - \frac{1}{n} \to a$ as $n \to \infty$ so $a \leq x < b \implies x \in (a, b)$.

            Therefore, $\bigcap_{n=1}^{\infty} [a - \frac{1}{n}, b) = (a, b)$.
        \end{proof}

        But we can write any interval $[a, b)$ by 
        \[[a, b) = [a, \infty) \cup [b, \infty)^c\]

        So any open set in $\R$ is a countable union of sets in $\Ec$ (and their complements). 

        Hence, $\B_{\R} \sub \M(\Ec)$.

        The argument for $(-\infty, a]$ is similar with 
        \[(a, b) = \bigcap_{n=1}^{\infty} (a, b + \frac{1}{n}]\]
            
    \color{black}

\end{itemize}

\pagebreak 

3. If $(X, \M, \mu)$ is a measure space and $E, F \in \M$, then 
\[\mu(E) + \mu(F) = \mu(E \cup F) + \mu(E \cap F)\]

    \color{blue}
        First notice that 
        \begin{align*}
            E &= (E \setminus F) \cup (E \cap F)\\
            F &= (F \setminus E) \cup (E \cap F)
        \end{align*}
        which are each disjoint unions.

        So 
        \begin{align*}
            \mu(E) + \mu(F) &= \mu((E \setminus F) \cup (E \cap F)) + \mu((F \setminus E) \cup (E \cap F))\\
            &= \mu(E \setminus F) + \mu(E \cap F) + \mu(F \setminus E) + \mu(E \cap F)\\
            &= \mu(E \cap F) + \mu((E \setminus F) \cup (F \setminus E) \cup (E \cap F))\\
        \end{align*}

        But 
        \begin{align*}
            (E \setminus F) \cup (F \setminus E) &= (E \cap F^c) \cup (F \cap E^c)\\ 
                &= [(E \cup F) \cap (E \cup E^c)] \cap [(E \cup E^c) \cap (F^c \cup E^c)]\\
                &= (E \cup F) \cap (F^c \cup E^c)\\ 
                &= (E \cup F) \cap (E \cap F)^c\\
        \end{align*}

        So 
        \begin{align*}
            \mu((E \setminus F) \cup (F \setminus E) \cup (E \cap F)) &= \mu((E \cup F) \cap (E \cap F)^c \cup (E \cap F))\\ 
            &= \mu((E \cup F) \cap X)\\ 
            &= \mu(E \cup F)            
        \end{align*}

        Therefore, 
        \[\mu(E) + \mu(F) = \mu(E \cap F) + \mu(E \cup F) \qed\]
    \color{black}

\pagebreak

4. Let $(X, \M, \mu)$ be a measure space and $\{E_j\}_{j=1}^{\infty} \sub \M$, then 
\[\mu(\liminf E_j) \leq \liminf \mu(E_j)\]

Also, if $\mu(\bigcup_{j=1}^{\infty} E_j) < \infty$, then
\[ \mu(\limsup E_j) \geq \limsup \mu(E_j)\]

    \color{blue}
        Consider $\mu(\liminf E_j)$. By definition,
        \[\mu(\liminf E_j) = \mu\left(\bigcup_{k=1}^\infty \bigcap_{j=k}^{\infty} E_j\right)\]

        Let 
        \[F_k = \bigcap_{j=k}^{\infty} E_j\]
        so $F_1 \sub F_2 \sub \dots$. 

        By continuity from below, 
        \[\mu(\liminf E_j) = \mu\left(\bigcup_{k=1}^\infty F_k\right) = \lim_{k \to \infty} \mu(F_k) = \lim_{k\to \infty} \mu\left(\bigcap_{j = k}^{\infty} E_j\right) \]

        But for any $n \geq k$, $\bigcap_{j = n}^{\infty} E_j \sub E_n$ so by monotonicity, 
        \[\mu\left(\bigcap_{j = n}^{\infty} E_n\right) \leq \mu(E_n)\]

        And indeed it suffices to choose the smallest: 
        \[\mu\left(\bigcap_{j = k}^{\infty} E_j\right) \leq \inf_{j \geq k} \mu(E_j)\]

        Therefore, 
        \[\mu(\liminf E_j) \leq \lim_{k \to \infty} \inf_{j \geq k} \mu(E_k) = \liminf \mu(E_j)\]

        \div 

        Now suppose $\mu(\bigcup_{j=1}^{\infty} E_j) < \infty$. As before, 
        \begin{align*}
            \mu(\limsup E_j) &= \mu\left(\bigcap_{k=1}^{\infty} \bigcup_{j=k}^{\infty} E_j\right)\\ 
            &= \mu\left(\bigcap_{k=1}^{\infty} F_k\right)\\ 
            &= \lim_{k \to \infty} \mu(F_k) \qquad (\text{Continuity from above since } \mu(F_1) < \infty)\\ 
            &= \lim_{k \to \infty} \mu\left(\bigcup_{j=k}^{\infty} E_j\right)\\ 
            &\geq \lim_{k \to \infty} \sup_{j \geq k} \mu(E_j) \qquad (\text{Monotonicity})\\
            &= \limsup \mu(E_j) \qed
        \end{align*}

    \color{black}


\pagebreak

5. Let $\mu^*$ be an outer measure. Let $\{E_k\}_{k=1}^{\infty}$ be a sequence of sets such that 
\[\sum_{k=1}^\infty \mu^*(E_k) < \infty\]
show that $\mu^*(\limsup E_k) = 0$

    \color{blue}
        Certainly $\mu^*(\limsup E_k) \geq 0$. We will seek to further show that $\mu^*(\limsup E_k) \leq 0$.

        On the contrary, suppose $\mu^*(\limsup E_k) = m > 0$. 
        
        By definition of $\limsup$, 
        \[\mu^*(\limsup E_k) = \mu^*\left(\bigcap_{n=1}^{\infty} \bigcup_{k=n}^{\infty} E_k\right)\]
        
        For notational convenience, let $F_n = \bigcup_{k=n}^{\infty} E_k$. Then $F_1 \supseteq F_2 \supseteq \dots$. 
        
        Then 
        \[\mu^*(\limsup E_k) = \mu^* \left(\bigcap_{n=1}^{\infty} F_n\right) = m\]

        Note however, that any element $x \in \bigcap_{n=1}^{\infty} F_n$ is in $\bigcup_{k=n}^{\infty} E_k$ for infinitely many $n$ by definition of $F_n$. Hence, 
        \[\bigcap_{n=1}^{\infty} F_n \sub \bigcup_{k=n}^{\infty} E_k\]
        so by monotonicity, 
        \[\mu^* \left(\bigcap_{n=1}^{\infty} F_n\right) = m \leq \mu^*\left(\bigcup_{k=n}^{\infty} E_k\right)\]

        \textbf{Lemma:} If $\mu^*$ is an outer measure and $\{E_k\}_1^{\infty}$ a sequence of sets, 
        \[\mu^*(\bigcup_{k=n}^{\infty} E_k) \leq \sum_{k=n+1}^{\infty} \mu^*(E_k)\]

        \begin{proof}
            \emph{Proof:} Define the sequence of sets $\{F_k\}_1^{\infty}$ by $F_k = E_{n+k}$. This is still a countably infinite sequence of sets in $\M$ so by subadditivity,
            \[\mu^*\left(\bigcup_{k=n+1}^\infty E_k\right) = \mu^*\left(\bigcup_{k=1}^\infty F_k\right) \leq \sum_{k=1}^{\infty} \mu^*(F_k) = \sum_{k=n+1}^{\infty} \mu^*(E_k)\]
        \end{proof}

        By assumption, $\sum_{k=1}^{\infty} \mu^*(E_k) < \infty$ so it must converge to a finite value, say $S$. Let $S_n$ be its sequence of partial sums. By definition of series convergence, $\forall \ep > 0$, $\exists N \in \N$ such that $n \geq N$ implies 
        \[\abs{S - S_n} < \ep\]

        Choose $\ep = \frac{m}{2}$. Then $\exists N \in \N$ such that $n \geq N$ implies
        \[\abs{S - S_n} = \sum_{k=1}^{\infty} \mu^*(E_k) - \sum_{k=1}^n \mu^*(E_k) = \sum_{k={n+1}}^{\infty} \mu^*(E_k) < \frac{m}{2}\]

        But by the Lemma, 
        \[\mu^*(\limsup E_k) = m \leq \mu^*\left(\bigcup_{k=n+1}^{\infty} E_k\right) \leq \sum_{k={n+1}}^{\infty} \mu^*(E_k) < \frac{m}{2}\]  
        
        And $0 < m < \frac{m}{2}$ is a contradiction, so $\mu^*(\limsup E_k) = 0$,


    \color{black}


\end{document}