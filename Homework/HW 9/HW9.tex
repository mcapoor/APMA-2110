\documentclass[12pt]{article}
\usepackage[utf8]{inputenc}
\usepackage{geometry}
\geometry{letterpaper, margin=0.5in}
\usepackage{graphicx} 
\usepackage{parskip}
\usepackage{booktabs}
\usepackage{array} 
\usepackage{paralist} 
\usepackage{verbatim}
\usepackage{subfig}
\usepackage{fancyhdr}
\usepackage{sectsty}
\usepackage[shortlabels]{enumitem}

\pagestyle{fancy}
\renewcommand{\headrulewidth}{0pt} 
\lhead{}\chead{}\rhead{}
\lfoot{}\cfoot{\thepage}\rfoot{}

%%% ToC (table of contents) APPEARANCE
\usepackage[nottoc,notlof,notlot]{tocbibind} 
\usepackage[titles,subfigure]{tocloft}
\renewcommand{\cftsecfont}{\rmfamily\mdseries\upshape}
\renewcommand{\cftsecpagefont}{\rmfamily\mdseries\upshape} %

\usepackage{amsmath}
\usepackage{amssymb}
\usepackage{mathtools}
\usepackage{empheq}
\usepackage{xcolor}
\usepackage{bbm}
\usepackage{tikz}
\usepackage{pgfplots}
\usepackage{tikz-cd}
\pgfplotsset{compat=1.18}

\newcommand{\ans}[1]{\boxed{\text{#1}}}
\newcommand{\vecs}[1]{\langle #1\rangle}
\renewcommand{\hat}[1]{\widehat{#1}}

\renewcommand{\P}{\mathbb{P}}
\newcommand{\R}{\mathbb{R}}
\newcommand{\E}{\mathbb{E}}
\newcommand{\Z}{\mathbb{Z}}
\newcommand{\N}{\mathbb{N}}
\newcommand{\Q}{\mathbb{Q}}
\newcommand{\C}{\mathbb{C}}

\newcommand{\ind}{\mathbbm{1}}
\newcommand{\qed}{\quad \blacksquare}

\newcommand{\brak}[1]{\left\langle #1 \right\rangle}
\newcommand{\bra}[1]{\left\langle #1 \right\vert}
\newcommand{\ket}[1]{\left\vert #1 \right\rangle}

\newcommand{\abs}[1]{\left\vert #1 \right\vert}
\newcommand{\mfX}{\mathfrak{X}}
\newcommand{\ep}{\varepsilon}

\newcommand{\Ec}{\mathcal{E}}
\newcommand{\A}{\mathcal{A}}
\newcommand{\F}{\mathcal{F}}
\newcommand{\Cc}{\mathcal{C}}
\newcommand{\B}{\mathcal{B}}
\newcommand{\M}{\mathcal{M}}
\newcommand{\X}{\chi}
\renewcommand{\L}{\mathcal{L}}

\newcommand{\sub}{\subseteq}
\newcommand{\st}{\text{ s.t. }}
\newcommand{\card}{\text{card }}
\renewcommand{\div}{\vspace*{10pt}\hrule\vspace*{10pt}}
\newcommand{\surj}{\twoheadrightarrow}
\newcommand{\inj}{\hookrightarrow}
\newcommand{\biject}{\hookrightarrow \hspace{-8pt} \rightarrow}
\renewcommand{\bar}[1]{\overline{#1}}
\newcommand{\overcirc}[1]{\overset{\circ}{#1}}
\newcommand{\diam}{\text{diam }}

\renewcommand{\Re}{\text{Re}\,}
\renewcommand{\Im}{\text{Im}\,}
\newcommand{\sign}{\text{sign}\,}

\newcommand*{\tbf}[1]{\ifmmode\mathbf{#1}\else\textbf{#1}\fi}

\usepackage{tcolorbox}
\tcbuselibrary{breakable, skins}
\tcbset{enhanced}
\newenvironment*{tbox}[2][gray]{
    \begin{tcolorbox}[
        parbox=false,
        colback=#1!5!white,
        colframe=#1!75!black,
        breakable,
        title={#2}
    ]}
    {\end{tcolorbox}}

\newenvironment*{exercise}[1][red]{
    \begin{tcolorbox}[
        parbox=false,
        colback=#1!5!white,
        colframe=#1!75!black,
        breakable
    ]}
    {\end{tcolorbox}}

\newenvironment*{proof}[1][blue]{
\begin{tcolorbox}[
    parbox=false,
    colback=#1!5!white,
    colframe=#1!75!black,
    breakable
]}
{\end{tcolorbox}}

\title{APMA 2110 - Homework 9}
\author{Milan Capoor}
\date{Nov 18, 2024}

\begin{document}
\maketitle
1. Let $f \in \L^1(\R)$ be Lebesgue integrable. Prove that for any $\ep > 0$, $\exists \delta > 0$ such that if $\abs{h} < \delta$, then 
\[\int_{\R} \abs{f(x + h) - f(x)}\; dx < \ep\]

    \color{blue}
        By integrability, pick $R_1 > 0$ such that 
        \[\int_{-\infty}^{-R} \abs{f(x+ h)}\; dx < \frac{\ep}{6}, \quad \int_{R}^{\infty} \abs{f(x+ h)} < \frac{\ep}{6}\]

        Similarly, pick $R_2 > 0$ such that
        \[\int_{-\infty}^{-R} \abs{f(x)}\; dx < \frac{\ep}{6}, \quad \int_{R}^{\infty} \abs{f(x)} < \frac{\ep}{6}\]

        Let $R = \max\{R_1, R_2\}$. Then,
        \begin{align*}
            \int_{\R} \abs{f(x+ h) -f(x)}\; dx &= \int_{-\infty}^{-R} \abs{f(x+ h) - f(x)}\; dx + \int_{-R}^{R} \abs{f(x+ h) - f(x)}\; dx + \int_{R}^{\infty} \abs{f(x+ h) - f(x)}\; dx \\
            &\leq \int_{-\infty}^{-R} \abs{f(x+ h)} + \abs{f(x)} \;dx + \int_{-R}^{R} \abs{f(x+ h) - f(x)}\; dx + \int_{R}^{\infty} \abs{f(x+ h)} + \abs{f(x)}\; dx \\
            &< \frac{2\ep}{3} + \int_{-R}^{R} \abs{f(x+ h) - f(x)}\; dx 
        \end{align*}

        It suffices to show that 
        \[\int_{-R}^{R} \abs{f(x+ h) - f(x)}\; dx < \frac{\ep}{3}\]

        By reduction to smooth functions, $\exists \phi$ such that 
        \[\int_{-R}^{R+h} \abs{f(x) - \phi(x)}\; dx < \frac{\ep}{9}\]
        so 
        \[\int_{-R}^R \abs{f(x + h) - f(x)}\; dx \leq \int_{-R}^R \abs{f(x + h) - \phi(x + h)} + \int_{-R}^R \abs{\phi(x+ h) - \phi(x)} \; dx + \int_{-R}^R \abs{\phi(x) - f(x)}\; dx\]
        so $\int_{-R}^R \abs{f(x+ h) - \phi(x + h)} \; dx < \frac{\ep}{9}$ and $\int_{-R}^R \abs{\phi(x) - f(x)}\; dx < \frac{\ep}{9}$. 

        All that is left is to show that 
        \[\int_{-R}^R \abs{\phi(x + h) - \phi(x)}\; dx < \frac{\ep}{9}\]

        But $\phi$ is smooth, hence continuous: $\exists \delta > 0$ s.t. $\abs{x - y} < \delta \implies \abs{\phi(x) - \phi(y)} < \frac{\ep}{18R}$.

        In particular, 
        \[\abs{x + h - x} = \abs{h} < \delta \implies \abs{\phi(x + h) - \phi(x)} < \frac{\ep}{18R}\]
        so by monotonicity,
        \[\int_{-R}^R \abs{\phi(x + h) - \phi(x)}\; dx \leq \int_{-R}^R \frac{\ep}{18R} \; dx = \frac{R\ep}{18R} + \frac{R\ep}{18R} = \frac{\ep}{9}\]

        All together, for $\abs{h} < \delta$,
        \[\int_{-R}^{R} \abs{f(x+ h) - f(x)}\; dx \leq \frac{\ep}{9} + \frac{\ep}{9} + \frac{\ep}{9} = \frac{\ep}{3}\] 
        and 
        \[\int_{\R} \abs{f(x+ h) -f(x)}\; dx < \frac{2\ep}{3} + \frac{\ep}{3} = \ep \qed\]
    \color{black}
  
\pagebreak

2. Let $f \in \L^1(\R)$ be Lebesgue integrable. Show that 
\[F(t) = \int_{-\infty}^x f(t)\; dt\]
is continuous. 

    \color{blue}
        Let $\ep > 0$ and suppose $\abs{x - y} < \delta$ for $\delta > 0$. We want to show that 
        \[\abs{F(x) - F(y)} = \abs{\int_{-\infty}^x f(t)\; dt - \int_{-\infty}^y f(t)\; dt} = \abs{\int_y^x f(t)\; dt} < \ep\]

        Since $f \in \L^1$, from a lemma in class, 
        \[\abs{\int_y^x f(t)\; dt} \leq \int_y^x \abs{f(t)}\; dt\]

        By reduction to smooth functions, $\exists \phi$ such that
        \[\int_y^x \abs{f(t) - \phi(t)}\; dt < \frac{\ep}{2} \implies \int_y^x \abs{f(t)}\; dt \leq \int_y^x \abs{f(t) - \phi(t)}\; dt + \int_y^x \abs{\phi(t)}\; dt\]

        Since $\phi$ is smooth, it is finite on the closed interval $[y, x]$ (otherwise, not continuous). Let $\abs{\phi(t)} \leq M$ for $t \in [y, x]$. Then,
        \[\int_y^x \abs{f(t)} \;dt \leq M\abs{x - y}\] 

        Let $\delta = \frac{\ep}{2M}$. Again by continuity (smoothness), 
        \[\abs{x - y} < \delta = \frac{\ep}{2M} \implies \int_y^x \abs{f(t)}\; dt < \frac{\ep}{2}\] 

        So for $\abs{x - y} < \frac{\ep}{2M}$,
        \[\abs{F(x) - F(y)} < \ep\]
        and $F$ is continuous. $\qed$

    \color{black}

\pagebreak 

3. Let $X = Y = [0, 1]$, $\A = \B[0, 1]$ (Borel Sets), $\mu$ be the Lebesgue measure, and $\nu$ be the counting measure. If $D = \{(x, x): x \in [0, 1]\}$ is the diagonal in $X \times Y$, show 
\begin{itemize}
    \item $\int \int \ind_D \; d\mu \; d\nu$
    \item $\int \int \ind_D \; d\nu \; d\mu$
    \item $\int \ind_D \; d(\mu \times \nu)$
\end{itemize} 
are all unequal. 

    \color{blue}
        First, notice that $D \sub \B([0, 1]) \otimes \B([0, 1])$ so by a Lemma from class, 
        \begin{align*}
            D_x &= \{y \in Y: (x, y) \in D\} \in \B([0, 1])\\
            D^y &= \{x \in X: (x, y) \in D\} \in \B([0, 1])
        \end{align*}
        so
        \begin{align*}
            \int \int \ind_{D(x, y)} \; d\mu(x) \; d\nu(y) &= \int \left(\int \ind_{D^y(x)}\; d\mu\right)\; d\nu\\ 
                &= \int \left(\int_{\{x: x = y\}} 1 \; d\mu\right) \; d\nu\\ 
                &= \int \mu(\{y\}) \; d\nu = 0
        \end{align*}

        Similarly,
        \begin{align*}
            \int \left(\int \ind_D \; d\nu(y)\right)\; d\mu(x) &= \int \left(\int_D \ind_{D_x} \; d\nu(y)\right)\; d\mu(x)\\ 
            &= \int \nu(\{x\}) \; d\mu \\
            &= \int 1 \; d\mu\\ 
            &= \mu([0, 1]) = 1
        \end{align*}

        Define 
        \[(\mu \times \nu)(E) = \inf\left\{\sum_i \mu(A_i) \nu(B_i) \; \bigg\vert \; E \sub \bigcup_i A_i \times B_i\right\}\]
        for $A_i \times B_i$ disjoint rectangles. 
        
        If $D \sub \bigcup_i A_i \times B_i$, then $\bigcup_i (A_i \cap B_i)$ covers $[0, 1]$ and we must have that $\mu(A_n \cap B_n) > 0$ for some $n$. But this implies that $A_n \cap B_n$ is uncountable, so $\nu(A_n \cap B_n) = \infty$. Hence,
        \[\sum_i \mu(A_n)\nu(B_n) = \infty \]
        for all covers of $D$ by rectangles so $(\mu \times \nu)(D) = \infty$.

        Hence, 
        \begin{align*}
            \int \ind_D \; d(\mu \times \nu) = \infty\\ 
            \int \int \ind_D \; d\mu \; d\nu = 0\\
            \int \int \ind_D \; d\nu \; d\mu = 1 \qed
        \end{align*}
       
    \color{black}


\pagebreak 

4. (Fubini) Let $(X, \A, \mu)$ and $(Y, \B, \nu)$ be $\sigma$-finite measure spaces and let $(X \times Y, \L, \mu \times \nu)$ be the product measure. If $f \in \L^1(\mu \times \nu)$, show that 
\begin{enumerate}
    \item $f_x$ is $\B$-measurable for a.e. $x$ 
    \item $f^y$ is $\A$-measurable for a.e. $y$
    \item $f_x$ is integrable for a.e. $x$
    \item $f^y$ is integrable for a.e. $y$
    \item If $x \to \int f_x \; d\nu$ and $y \to \int f^y \; d\mu$ are measurable and integrable, then
    \[\int f \;d(\mu \times \nu) = \int \int f(x, y) \; d\mu(x)\,d\nu(y) = \int \int f(x, y) \; d\nu(y)\,d\mu(x)\]
\end{enumerate}

    \color{blue}
        Since $f \in \L^1(\mu \times \nu)$, it is measurable on $\M \otimes \N$. By a lemma in class, we have that $f_x$ and $f_y$ are measurable for a.e. $x$ and $y$ respectively.

        In class, we showed the result for the special case $f = \ind_E$ for $E \in \M \times \N$. By linearity, the result holds for simple functions. For $f \in \L^1(\mu \times \nu)$, we can sat $f \geq 0$ WLOG and then approximate $f$ by simple functions $\phi_n \nearrow f$. 

        Let 
        \begin{align*}
            g(x) &= \int f_x\; d\nu\\ 
            h(y) &= \int f^y\; d\mu
        \end{align*}
        and $g_n, h_n$ be the corresponding sections of $\phi_n$.

        By MCT, $g_n \nearrow g$ and $h_n \nearrow h$ so $g, h$ are measurable and 
        \begin{align*}
            \int g \; d\mu &= \lim \int g_n \; d\mu = \lim \int \phi_n \; d(\mu \times \nu) = \int f \; d(\mu \times \nu)\\
            \int h \; d\nu &= \lim \int h_n \; d\nu = \lim \int \phi_n \; d(\mu \times \nu) = \int f \; d(\mu \times \nu)
        \end{align*}

        In particular, since $f \in \L^1(\mu \times \nu)$, $f_x$ and $f^y$ are integrable for a.e. $x$ and $y$ respectively and the result follows. $\qed$


    \color{black}

\pagebreak

5. Let $\nu$ be a measure on the Borel sets of the positive real line $[0, \infty)$ such that 
\[\phi(t) = \nu([0, t))\]

Let $(X, \M, \mu)$ be a measure space and $f \geq 0$ measurable. Show that 
\[\int_{\Omega} \phi(f(x)) \; d\mu = \int_0^{\infty} \mu(\{x: f(x) > t\})\; dt\]

    \color{blue} 
        \begin{align*}
            \int_{X} \phi(f(x)) \; d\mu &= \int_{X} \nu[0, f(x))\; d\mu\\ 
            &= \int_X \int_0^{\infty} \ind_{[0, f(x))} \; d\nu \; d\mu\\ 
            &= \int_0^{\infty} \int_X \ind_{[0, f(x))} \; d\mu \; d\nu \qquad (\text{Tonelli})\\ 
            &= \int_0^{\infty} \mu([0,f(x))) \; d\nu\\
            &= \int_0^{\infty} \mu(\{t: f(x) > t\})\; d\nu\\ 
            &= \int_0^{\infty} \mu(\{x: f(x) > t\})\; dt
        \end{align*}
    \color{black}


\end{document}
